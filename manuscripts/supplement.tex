\documentclass[]{components/elsarticle}
\usepackage[T1]{fontenc}
\usepackage{lmodern}
\usepackage{amssymb,amsmath}
\usepackage{ifxetex,ifluatex}
\usepackage{fixltx2e} % provides \textsubscript
% use upquote if available, for straight quotes in verbatim environments
\IfFileExists{upquote.sty}{\usepackage{upquote}}{}
\ifnum 0\ifxetex 1\fi\ifluatex 1\fi=0 % if pdftex
  \usepackage[utf8]{inputenc}
\else % if luatex or xelatex
  \ifxetex
    \usepackage{mathspec}
    \usepackage{xltxtra,xunicode}
  \else
    \usepackage{fontspec}
  \fi
  \defaultfontfeatures{Mapping=tex-text,Scale=MatchLowercase}
  \newcommand{\euro}{€}
\fi
% use microtype if available
\IfFileExists{microtype.sty}{\usepackage{microtype}}{}
\usepackage[margin=1in]{geometry}
\usepackage{longtable,booktabs}
\usepackage{graphicx}
% Redefine \includegraphics so that, unless explicit options are
% given, the image width will not exceed the width of the page.
% Images get their normal width if they fit onto the page, but
% are scaled down if they would overflow the margins.
\makeatletter
\def\ScaleIfNeeded{%
  \ifdim\Gin@nat@width>\linewidth
    \linewidth
  \else
    \Gin@nat@width
  \fi
}
\makeatother
\let\Oldincludegraphics\includegraphics
{%
 \catcode`\@=11\relax%
 \gdef\includegraphics{\@ifnextchar[{\Oldincludegraphics}{\Oldincludegraphics[width=\ScaleIfNeeded]}}%
}%
\ifxetex
  \usepackage[setpagesize=false, % page size defined by xetex
              unicode=false, % unicode breaks when used with xetex
              xetex]{hyperref}
\else
  \usepackage[unicode=true]{hyperref}
\fi
\hypersetup{breaklinks=true,
            bookmarks=true,
            pdfauthor={},
            pdftitle={Supplement for: Avoiding tipping points in fisheries management through Gaussian Process Dynamic Programming},
            colorlinks=true,
            citecolor=blue,
            urlcolor=blue,
            linkcolor=magenta,
            pdfborder={0 0 0}}
\urlstyle{same}  % don't use monospace font for urls
\setlength{\parindent}{0pt}
\setlength{\parskip}{6pt plus 2pt minus 1pt}
\setlength{\emergencystretch}{3em}  % prevent overfull lines
\setcounter{secnumdepth}{0}

%%% Change title format to be more compact
\usepackage{titling}
\setlength{\droptitle}{-2em}
  \title{Supplement for: Avoiding tipping points in fisheries management through
Gaussian Process Dynamic Programming}
  \pretitle{\vspace{\droptitle}\centering\huge}
  \posttitle{\par}
  \author{true \and true \and true}
  \preauthor{\centering\large\emph}
  \postauthor{\par}
  \date{}
  \predate{}\postdate{}




\begin{document}

\maketitle


\tableofcontents

\appendix
\renewcommand*{\thefigure}{S\arabic{figure}}
\renewcommand*{\thetable}{S\arabic{table}} \setcounter{figure}{0}
\setcounter{table}{0}

\section{Code}\label{code}

All code used in producing this analysis has been embedded into the
manuscript sourcefile using the Dynamic Documentation tool,
\texttt{knitr} for the R language (Xie 2013), available at
\href{https://github.com/cboettig/nonparametric-bayes/}{github.com/cboettig/nonparametric-bayes/}

To help make the mathematical and computational approaches presented
here more accessible, we provide a free and open source (MIT License) R
package that implements the GPDP process as it is presented here. Users
should note that at this time, the R package has been developed and
tested explicitly for this analysis and is not yet intended as a general
purpose tool. The manuscript source-code described above illustrates how
these functions are used in this analysis. This package can be installed
following the directions above.

\subsection{Dependencies \&
Reproducibility}\label{dependencies-reproducibility}

The code provided should run on any common platform (Windows, Mac or
Linux) that has R and the necessary R packages installed (including
support for the jags Gibbs sampler). The DESCRIPTION file of our R
package, \texttt{nonparametricbayes}, lists all the software required to
use these methods. Additional software requirements for the other
comparisons shown here, such as the Gibbs sampling for the parametric
models, are listed under the Suggested packages list.

Nonetheless, installing the dependencies needed is not a trivial task,
and may become more difficult over time as software continues to evolve.
To facilitate reuse, we also provide a Dockerfile and Docker image that
can be used to replicate and explore the analyses here by providing a
copy of the computational environment we have used, with all software
installed. Docker sofware (see \href{http://www.docker.com}{docker.com})
runs on most platforms as well as cloud servers. Use the command:

\begin{verbatim}
docker run -dP cboettig/nonparametric-bayes
\end{verbatim}

to launch an RStudio instance with the necessary software already
installed. See the Rocker-org project,
\href{https://github.com/rocker-org}{github.com/rocker-org} for more
detailed documentation on using Docker with R.

\section{Data}\label{data}

\subsection{Dryad Data Archive}\label{dryad-data-archive}

While the data can be regenerated using the code provided, for
convenience CSV files of the data shown in each graph are made available
on Dryad.

\subsection{Training data description}\label{training-data-description}

Each of our models $f(S_t)$ must be estimated from training data, which
we simulate from the Allen model with parameters $r = $ 2, $K =$ 8,
$C =$ 5, and $\sigma_g =$ 0.05 for $T=$ 40 timesteps, starting at
initial condition $X_0 = $ 5.5. The training data can be seen in Figure
1 and found in the table \texttt{figure1.csv}.

\subsection{Training data for sensitivity
analyses}\label{training-data-for-sensitivity-analyses}

A further 96 unique randomly generated training data sets are generated
for the sensitivity analysis, as described in the main text. The code
provided replicates the generation of these sets.

\section{Model performance outside the predicted range (Fig
S1)}\label{model-performance-outside-the-predicted-range-fig-s1}

Figure S1 illustrates the performance of the GP and parametric models
outside the observed training data. The mean trajectory under the
underlying model is shown by the black dots, while the corresponding
prediction made by the model shown by the box and whiskers plots.
Predictions are based on the true expected value in the previous time
step. Predicted distributions that lie entirely above the expected
dynamics indicate the expectation of stock sizes higher than what is
actually expected. The models differ both in their expectations and
their uncertainty (colored bands show two standard deviations away).
Note that the GP is particularly uncertain about the dynamics relative
to structurally incorrect models like the Ricker.

\begin{figure}[htbp]
\centering
\includegraphics{components/figure/supplement-figure_S1-1.pdf}
\caption{Outside the range of the training data (Figure 1), the true
dynamics (black dots) fall outside the uncertainty (two standard
deviations, colored bands) of the structurally incorrect parametric
models (Ricker, Myers), but inside the uncertainty predicted by the GP.
Points show the stock size simulated by the true model. Overlay shows
the range of states predicted by each model, based on the state observed
in the previous time step. The Ricker model always (wrongly) predicts
positive population growth, while the actual population shrinks in each
step as the initial condition falls below the Allee threshold of the
underlying model (Allen). Note that because it does not assume a
parametric form but instead relies more directly on the data, the GP is
both more pessimistic and more uncertain about the future state than the
parametric models.}
\end{figure}

\newpage

\section{Further sensitivity analysis (Fig
S2)}\label{further-sensitivity-analysis-fig-s2}

The Latin hypercube approach systematically varies all combinations of
parameters, providing a more general test than varying only one
parameter at a time. We loop across eight replicates of three different
randomly generated parameter sets for each of two different generating
models (Allen and Myers) over two different noise levels (0.01 and
0.05), for a total of 8 x 3 x 2 x 2 = 96 scenarios. The Gaussian Process
performs nearly optimally in each case, relative to the optimal solution
with no parameter or model uncertainty (Figure S10, appendix).

\begin{figure}[htbp]
\centering
\includegraphics{components/figure/supplement-figure_S2-1.pdf}
\caption{Sensitivity Analysis. Histograms shows the ratio of the
realized net present value derived when managing under the GPDP over the
optimal value given the true model and true parameters. Values of 1
indicate optimal performance. Columns indicate different models, rows
different noise levels, and colors indicate the parameter set used.
Grouped over stochastic replicates applying the contol policy and
stochastic replicates of training data generated from the model
indicated, see raw data for details. Randomly chosen parameter values
for the models shown in tables below.}
\end{figure}

\begin{longtable}[c]{@{}cccc@{}}
\toprule\addlinespace
\begin{minipage}[b]{0.15\columnwidth}\centering
~
\end{minipage} & \begin{minipage}[b]{0.07\columnwidth}\centering
r
\end{minipage} & \begin{minipage}[b]{0.07\columnwidth}\centering
K
\end{minipage} & \begin{minipage}[b]{0.09\columnwidth}\centering
theta
\end{minipage}
\\\addlinespace
\midrule\endhead
\begin{minipage}[t]{0.15\columnwidth}\centering
\textbf{set.A}
\end{minipage} & \begin{minipage}[t]{0.07\columnwidth}\centering
1.103
\end{minipage} & \begin{minipage}[t]{0.07\columnwidth}\centering
7.949
\end{minipage} & \begin{minipage}[t]{0.09\columnwidth}\centering
2.288
\end{minipage}
\\\addlinespace
\begin{minipage}[t]{0.15\columnwidth}\centering
\textbf{set.B}
\end{minipage} & \begin{minipage}[t]{0.07\columnwidth}\centering
1.485
\end{minipage} & \begin{minipage}[t]{0.07\columnwidth}\centering
9.775
\end{minipage} & \begin{minipage}[t]{0.09\columnwidth}\centering
3.524
\end{minipage}
\\\addlinespace
\bottomrule
\addlinespace
\caption{Randomly chosen parameter sets for the Allen models in Figure
S2.}
\end{longtable}

\begin{longtable}[c]{@{}cccc@{}}
\toprule\addlinespace
\begin{minipage}[b]{0.15\columnwidth}\centering
~
\end{minipage} & \begin{minipage}[b]{0.07\columnwidth}\centering
r
\end{minipage} & \begin{minipage}[b]{0.07\columnwidth}\centering
K
\end{minipage} & \begin{minipage}[b]{0.07\columnwidth}\centering
C
\end{minipage}
\\\addlinespace
\midrule\endhead
\begin{minipage}[t]{0.15\columnwidth}\centering
\textbf{set.C}
\end{minipage} & \begin{minipage}[t]{0.07\columnwidth}\centering
1.769
\end{minipage} & \begin{minipage}[t]{0.07\columnwidth}\centering
10.46
\end{minipage} & \begin{minipage}[t]{0.07\columnwidth}\centering
4.301
\end{minipage}
\\\addlinespace
\begin{minipage}[t]{0.15\columnwidth}\centering
\textbf{set.D}
\end{minipage} & \begin{minipage}[t]{0.07\columnwidth}\centering
2.075
\end{minipage} & \begin{minipage}[t]{0.07\columnwidth}\centering
10.95
\end{minipage} & \begin{minipage}[t]{0.07\columnwidth}\centering
4.915
\end{minipage}
\\\addlinespace
\bottomrule
\addlinespace
\caption{Randomly chosen parameter sets for the Myers models in Figure
S2.}
\end{longtable}

\newpage

\section{MCMC analyses}\label{mcmc-analyses}

This section provides figures and tables showing the traces from each of
the MCMC runs used to estimate the parameters of the models presented,
along with the resulting posterior distributions for each parameter.
Priors usually appear completely falt when shown against the posteriors,
but are summarized by tables the parameters of their corresponding
distributions for each case.

\subsection{GP MCMC (Fig S3-4)}\label{gp-mcmc-fig-s3-4}

\begin{figure}[htbp]
\centering
\includegraphics{components/figure/supplement-figure_S3-1.pdf}
\caption{Traces from the MCMC estimates of the GP model show reasonable
mixing (no trend) and sampling rejection rate (no piecewise jumps)}
\end{figure}

\begin{figure}[htbp]
\centering
\includegraphics{components/figure/supplement-figure_S4-1.pdf}
\caption{Posterior distributions from the MCMC estimate of the GP model.
Prior curves shown in red; note the posterior distributions are
significantly more peaked than the priors, showing that the data has
been informative and is not driven by the priors.}
\end{figure}

\newpage
\newpage

\subsection{Ricker Model MCMC (Fig
S5-6)}\label{ricker-model-mcmc-fig-s5-6}

\begin{figure}[htbp]
\centering
\includegraphics{components/figure/supplement-figure_S5-1.pdf}
\caption{Traces from the MCMC estimates of the Ricker model show
reasonable mixing (no trend) and sampling rejection rate (no piecewise
jumps). stdQ refers to the estimate of $\sigma$; deviance is -2 times
the log likelihood.}
\end{figure}

\begin{figure}[htbp]
\centering
\includegraphics{components/figure/supplement-figure_S6-1.pdf}
\caption{Posteriors from the MCMC estimate of the Ricker model. Note
that the model estimates a carrying capacity $K$ very close to the true
equilibrium where most of the observations were made, but is less
certain about the growth rate.}
\end{figure}

\begin{longtable}[c]{@{}ccc@{}}
\toprule\addlinespace
\begin{minipage}[b]{0.15\columnwidth}\centering
parameter
\end{minipage} & \begin{minipage}[b]{0.18\columnwidth}\centering
lower.bound
\end{minipage} & \begin{minipage}[b]{0.18\columnwidth}\centering
upper.bound
\end{minipage}
\\\addlinespace
\midrule\endhead
\begin{minipage}[t]{0.15\columnwidth}\centering
$r$
\end{minipage} & \begin{minipage}[t]{0.18\columnwidth}\centering
0.01
\end{minipage} & \begin{minipage}[t]{0.18\columnwidth}\centering
20
\end{minipage}
\\\addlinespace
\begin{minipage}[t]{0.15\columnwidth}\centering
$K$
\end{minipage} & \begin{minipage}[t]{0.18\columnwidth}\centering
0.01
\end{minipage} & \begin{minipage}[t]{0.18\columnwidth}\centering
40
\end{minipage}
\\\addlinespace
\begin{minipage}[t]{0.15\columnwidth}\centering
$\sigma$
\end{minipage} & \begin{minipage}[t]{0.18\columnwidth}\centering
1e-06
\end{minipage} & \begin{minipage}[t]{0.18\columnwidth}\centering
100
\end{minipage}
\\\addlinespace
\bottomrule
\addlinespace
\caption{Parameterization range for the uniform priors in the Ricker
model}
\end{longtable}

\newpage
\newpage 

\subsection{Myers Model MCMC (Fig
S7-8)}\label{myers-model-mcmc-fig-s7-8}

\begin{figure}[htbp]
\centering
\includegraphics{components/figure/supplement-figure_S7-1.pdf}
\caption{Traces from the MCMC estimates of the Myers model show
reasonable mixing (no trend) and sampling rejection rate (no piecewise
jumps). stdQ refers to the estimate of $\sigma$; deviance is -2 times
the log likelihood.}
\end{figure}

\begin{figure}[htbp]
\centering
\includegraphics{components/figure/supplement-figure_S8-1.pdf}
\caption{Posterior distributions from the MCMC estimates of the Myers
model. Note that with more free parameters, the posteriors reflect
greater uncertainty. In particular, the parameter $\theta$ includes
values both above 2, resulting in a tipping point, and below 2, where no
tipping point exists in the model. Though the dynamic program will
integrate over the full distribution, including those values
corresponding to tipping points, the weight of the model lies in the
region without tipping points.}
\end{figure}

\begin{longtable}[c]{@{}ccc@{}}
\toprule\addlinespace
\begin{minipage}[b]{0.15\columnwidth}\centering
parameter
\end{minipage} & \begin{minipage}[b]{0.18\columnwidth}\centering
lower.bound
\end{minipage} & \begin{minipage}[b]{0.18\columnwidth}\centering
upper.bound
\end{minipage}
\\\addlinespace
\midrule\endhead
\begin{minipage}[t]{0.15\columnwidth}\centering
$r$
\end{minipage} & \begin{minipage}[t]{0.18\columnwidth}\centering
1e-04
\end{minipage} & \begin{minipage}[t]{0.18\columnwidth}\centering
10
\end{minipage}
\\\addlinespace
\begin{minipage}[t]{0.15\columnwidth}\centering
$K$
\end{minipage} & \begin{minipage}[t]{0.18\columnwidth}\centering
1e-04
\end{minipage} & \begin{minipage}[t]{0.18\columnwidth}\centering
40
\end{minipage}
\\\addlinespace
\begin{minipage}[t]{0.15\columnwidth}\centering
$\theta$
\end{minipage} & \begin{minipage}[t]{0.18\columnwidth}\centering
1e-04
\end{minipage} & \begin{minipage}[t]{0.18\columnwidth}\centering
10
\end{minipage}
\\\addlinespace
\begin{minipage}[t]{0.15\columnwidth}\centering
$\sigma$
\end{minipage} & \begin{minipage}[t]{0.18\columnwidth}\centering
1e-06
\end{minipage} & \begin{minipage}[t]{0.18\columnwidth}\centering
100
\end{minipage}
\\\addlinespace
\bottomrule
\addlinespace
\caption{Parameterization range for the uniform priors in the Myers
model}
\end{longtable}

\newpage 
\newpage 

\subsection{Allen Model MCMC (Fig
S9-10)}\label{allen-model-mcmc-fig-s9-10}

\begin{figure}[htbp]
\centering
\includegraphics{components/figure/supplement-figure_S9-1.pdf}
\caption{Traces from the MCMC estimates of the Allen model show
reasonable mixing (no trend) and sampling rejection rate (no piecewise
jumps). stdQ refers to the estimate of $\sigma$; deviance is -2 times
the log likelihood.}
\end{figure}

\begin{figure}[htbp]
\centering
\includegraphics{components/figure/supplement-figure_S10-1.pdf}
\caption{Posteriors from the MCMC estimate of the Allen model. The Allen
model is the structurally correct model. Despite potential
identfiability issues in distinguishing between the stable and unstable
points ($K$ and $\theta$ respectively), the posterior estimates
successfully reflect both the the upper stable point ($K$), as well as
the significant probabilty of a tipping point ($\theta$) somewhere
between $K$ and extinction (0).}
\end{figure}

\begin{longtable}[c]{@{}ccc@{}}
\toprule\addlinespace
\begin{minipage}[b]{0.15\columnwidth}\centering
parameter
\end{minipage} & \begin{minipage}[b]{0.18\columnwidth}\centering
lower.bound
\end{minipage} & \begin{minipage}[b]{0.18\columnwidth}\centering
upper.bound
\end{minipage}
\\\addlinespace
\midrule\endhead
\begin{minipage}[t]{0.15\columnwidth}\centering
$r$
\end{minipage} & \begin{minipage}[t]{0.18\columnwidth}\centering
0.01
\end{minipage} & \begin{minipage}[t]{0.18\columnwidth}\centering
6
\end{minipage}
\\\addlinespace
\begin{minipage}[t]{0.15\columnwidth}\centering
$K$
\end{minipage} & \begin{minipage}[t]{0.18\columnwidth}\centering
0.01
\end{minipage} & \begin{minipage}[t]{0.18\columnwidth}\centering
20
\end{minipage}
\\\addlinespace
\begin{minipage}[t]{0.15\columnwidth}\centering
$X_C$
\end{minipage} & \begin{minipage}[t]{0.18\columnwidth}\centering
0.01
\end{minipage} & \begin{minipage}[t]{0.18\columnwidth}\centering
20
\end{minipage}
\\\addlinespace
\begin{minipage}[t]{0.15\columnwidth}\centering
$\sigma$
\end{minipage} & \begin{minipage}[t]{0.18\columnwidth}\centering
1e-06
\end{minipage} & \begin{minipage}[t]{0.18\columnwidth}\centering
100
\end{minipage}
\\\addlinespace
\bottomrule
\addlinespace
\caption{Parameterization range for the uniform priors in the Allen
model}
\end{longtable}

Xie, Y. 2013. Dynamic Documents with R and knitr. Chapman; Hall/CRCBoca
Raton, Florida.

\end{document}
