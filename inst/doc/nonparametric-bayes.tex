\documentclass[author-year, review]{elsarticle} %review=doublespace preprint=single 5p=2 column
%%% Begin My package additions %%%%%%%%%%%%%%%%%%%
\usepackage[hyphens]{url}
\usepackage{lineno} % add 
%\linenumbers % turns line numbering on 
\bibliographystyle{elsarticle-harv}
\biboptions{sort&compress} % For natbib
\usepackage{graphicx}
\usepackage{booktabs} % book-quality tables
%% Redefines the elsarticle footer
\makeatletter
\def\ps@pprintTitle{%
 \let\@oddhead\@empty
 \let\@evenhead\@empty
 \def\@oddfoot{\it \hfill\today}%
 \let\@evenfoot\@oddfoot}
\makeatother

% A modified page layout
\textwidth 6.75in
\oddsidemargin -0.15in
\evensidemargin -0.15in
\textheight 9in
\topmargin -0.5in
%%%%%%%%%%%%%%%% end my additions to header



\usepackage[T1]{fontenc}
\usepackage{lmodern}
\usepackage{amssymb,amsmath}
\usepackage{ifxetex,ifluatex}
\usepackage{fixltx2e} % provides \textsubscript
% use upquote if available, for straight quotes in verbatim environments
\IfFileExists{upquote.sty}{\usepackage{upquote}}{}
\ifnum 0\ifxetex 1\fi\ifluatex 1\fi=0 % if pdftex
  \usepackage[utf8]{inputenc}
\else % if luatex or xelatex
  \usepackage{fontspec}
  \ifxetex
    \usepackage{xltxtra,xunicode}
  \fi
  \defaultfontfeatures{Mapping=tex-text,Scale=MatchLowercase}
  \newcommand{\euro}{€}
\fi
% use microtype if available
\IfFileExists{microtype.sty}{\usepackage{microtype}}{}
\usepackage{color}
\usepackage{fancyvrb}
\newcommand{\VerbBar}{|}
\newcommand{\VERB}{\Verb[commandchars=\\\{\}]}
\DefineVerbatimEnvironment{Highlighting}{Verbatim}{commandchars=\\\{\}}
% Add ',fontsize=\small' for more characters per line
\newenvironment{Shaded}{}{}
\newcommand{\KeywordTok}[1]{\textcolor[rgb]{0.00,0.44,0.13}{\textbf{{#1}}}}
\newcommand{\DataTypeTok}[1]{\textcolor[rgb]{0.56,0.13,0.00}{{#1}}}
\newcommand{\DecValTok}[1]{\textcolor[rgb]{0.25,0.63,0.44}{{#1}}}
\newcommand{\BaseNTok}[1]{\textcolor[rgb]{0.25,0.63,0.44}{{#1}}}
\newcommand{\FloatTok}[1]{\textcolor[rgb]{0.25,0.63,0.44}{{#1}}}
\newcommand{\CharTok}[1]{\textcolor[rgb]{0.25,0.44,0.63}{{#1}}}
\newcommand{\StringTok}[1]{\textcolor[rgb]{0.25,0.44,0.63}{{#1}}}
\newcommand{\CommentTok}[1]{\textcolor[rgb]{0.38,0.63,0.69}{\textit{{#1}}}}
\newcommand{\OtherTok}[1]{\textcolor[rgb]{0.00,0.44,0.13}{{#1}}}
\newcommand{\AlertTok}[1]{\textcolor[rgb]{1.00,0.00,0.00}{\textbf{{#1}}}}
\newcommand{\FunctionTok}[1]{\textcolor[rgb]{0.02,0.16,0.49}{{#1}}}
\newcommand{\RegionMarkerTok}[1]{{#1}}
\newcommand{\ErrorTok}[1]{\textcolor[rgb]{1.00,0.00,0.00}{\textbf{{#1}}}}
\newcommand{\NormalTok}[1]{{#1}}
\usepackage{graphicx}
% We will generate all images so they have a width \maxwidth. This means
% that they will get their normal width if they fit onto the page, but
% are scaled down if they would overflow the margins.
\makeatletter
\def\maxwidth{\ifdim\Gin@nat@width>\linewidth\linewidth
\else\Gin@nat@width\fi}
\makeatother
\let\Oldincludegraphics\includegraphics
\renewcommand{\includegraphics}[1]{\Oldincludegraphics[width=\maxwidth]{#1}}
\ifxetex
  \usepackage[setpagesize=false, % page size defined by xetex
              unicode=false, % unicode breaks when used with xetex
              xetex]{hyperref}
\else
  \usepackage[unicode=true]{hyperref}
\fi
\hypersetup{breaklinks=true,
            bookmarks=true,
            pdfauthor={},
            pdftitle={Avoiding tipping points in the management of ecological systems: a non-parametric Bayesian approach},
            colorlinks=true,
            urlcolor=blue,
            linkcolor=magenta,
            pdfborder={0 0 0}}
\urlstyle{same}  % don't use monospace font for urls
\setlength{\parindent}{0pt}
\setlength{\parskip}{6pt plus 2pt minus 1pt}
\setlength{\emergencystretch}{3em}  % prevent overfull lines
\setcounter{secnumdepth}{0}
% Pandoc toggle for numbering sections (defaults to be off)
\setcounter{secnumdepth}{0}
% Pandoc header



\begin{document}
\begin{frontmatter}
  \title{Avoiding tipping points in the management of ecological systems: a
non-parametric Bayesian approach}
  \author[cstar]{Carl Boettiger\corref{cor1}}
  \cortext[cor1]{Corresponding author}
  \ead{cboettig@ucsc.edu}
  \author[cstar]{Marc Mangel}
  \author[noaa]{Stephan B. Munch}
  \address[cstar]{Center for Stock Assessment Research, Department of Applied Math and Statistics, University of California, Mail Stop SOE-2, Santa Cruz, CA 95064, USA}
  \address[noaa]{Southwest Fisheries Science Center, National Oceanic and Atmospheric Administration, 110 Shaffer Road, Santa Cruz, CA 95060, USA}
 \end{frontmatter}


\section{Abstract}\label{abstract}

Model uncertainty and limited data coverage are fundamental challenges
to robust ecosystem management. These challenges are acutely highlighted
by concerns that many ecological systems may contain tipping points.
Before a collapse, we do not know where the tipping points lie, if the
exist at all. Hence, we know neither a complete model of the system
dynamics nor do we have access to data in some large region of
state-space where such a tipping point might exist. These two sources of
uncertainty frustrate state-of-the-art parametric approaches to decision
theory and optimal control. I will illustrate how a non-parametric
approach using a Gaussian Process prior provides a more flexible
representation of this inherent uncertainty. Consequently, we can adapt
the Gaussian Process prior to a stochastic dynamic programming framework
in order to make robust management predictions under both model and
uncertainty and limited data.

\section{Introduction}\label{introduction}

Decision making under uncertainty is a ubiquitous challenge of natural
resource management and conservation.

The sudden collapse of fisheries and other ecosystems is an increasingly
widespread phenomenon and a pressing concern for ecological management
and conservation. Ecological dynamics are frequently complex and
difficult to measure, making uncertainty in our understanding a
prediction a persistent challenge to effective management.
Decision-theoretic approaches provide a framework to determine the best
sequence of actions in face of uncertainty, but only when that
uncertainty can be meaningfully quantified ({\textbf{???}}).

Uncertainty enters the decision-making process at many levels: intrinsic
stochasticity in biological processes, measurements, and implementation
of policy (\emph{e.g.} Reed 1979; C. W. Clark and Kirkwood 1986;
{\textbf{???}}; Sethi et al. 2005), parameteric uncertainty (\emph{e.g.}
Ludwig and Walters 1982; {\textbf{???}}; {\textbf{???}}; Schapaugh and
Tyre 2013), and model or structural uncertainty (\emph{e.g.} B. K.
Williams 2001; Cressie et al. 2009; Athanassoglou and Xepapadeas 2012).
Of these, structural uncertainty incorporates the least a priori
knowledge or assumptions and is generally the hardest to quantify.
Typical approaches assume a weak notion of uncertainty where a correct
or reasonable approximation of the dynamics must be identified from
among a handful of alternative models. Here we consider an approach that
addresses uncertainty at each of these levels without assuming the
dynamics follow a particular (i.e.~parametric) structure.

An additional source of uncertainty that has recieved less
attention\footnote{The concept of adaptive probing is one area that has
  explicitly addressed this kind of uncertainty, reaching rather
  opposite conclusions than what we observe here. Adaptive probing
  strategies follow from ``Dual Control'' or ``Active Adaptive
  Management'' approaches (e.g. Ludwig and Walters (1982)) that can
  trade off short term utility by choosing actions that can reduce
  uncertainty. Adaptive probing strategies arise when it is valuable to
  intentionally force a system far from the observed values even when
  the expected value such actions is low, as it provides much faster
  learning and consequent reduction of model uncertainty that can allow
  greater value to be derived later on. For instance, Ludwig and Walters
  (1982) show that it may be advantageous to fish an unexploited
  population very heavily at first to obtain a better estimate of the
  recruitment rate. This intuitive strategy when a population is
  governed by a Ricker or Beverton-Holt-like dynamic would clearly be
  disastrous if instead the dynamics contained an unforeseen tipping
  point. The best way to learn where the edge lies may be to walk up to
  it, but it is also the most dangerous.} arises when applying a
dynamical model outside the range of data on which it has been
estimated. This extrapolation uncertainty is felt most keenly in
decision-theoretic applications, as (a) exploring the potential action
space typically involves considering actions that may move the system
outside the range of observed behavior, and (b) decision-theoretic
alogrithms rely not only on reasonable estimates of the expected
outcomes, but depend on the weights given to all possible outcomes
(\emph{e.g.} Weitzman 2013).

Concerns over the potential for tipping points in ecological dynamics
(Scheffer et al. 2001) highlight the dangers of uncertainty in
ecological management and pose a substantial challenge to existing
decision-theoretic approaches (Brozović and Schlenker 2011). Because
intervention is often too late after a tipping point has been crossed
(but see Hughes et al. (2013)), management is most often concerned with
avoiding potentially catastrophic tipping points before any data is
available at or following a transition that would more clearly reveal
these regime shift dynamics (e.g. Bestelmeyer et al. 2012).

\emph{Map}

Here we illustrate how a stochastic dynamic programming (SDP) algorithm
(Mangel and Clark 1988; {\textbf{???}}) can be driven by the predictions
from a Bayesian non-parametric (BNP) approach (Munch, Kottas, and Mangel
2005). This provides two distinct advantages compared with contemporary
approaches. First, using a BNP sidesteps the need for an accurate
model-based description of the system dynamics. Second, the BNP can
better reflect uncertainty that arises when extrapolating a model
outside of the data on which it was fit. We illustrate that when the
correct model is not known, this latter feature is crucial to providing
a robust decision-theoretic approach in face of substantial structural
uncertainty.

This paper represents the first time the SDP decision-making framework
has been used without an a priori model of the underlying dynamics
through the use of the BNP approach. In contrast to parametric models
which can only reflect uncertainty in parameter estimates, the BNP
approach provides a more state-space dependent representation of
uncertainty. This permits a much greater uncertainty far from the
observed data than near the observed data. These features allow the
GP-SDP approach to find robust management solutions in face of limited
data and without knowledge of the correct model structure.

\begin{itemize}
\itemsep1pt\parskip0pt\parsep0pt
\item
  \emph{Note on ``not magic''}: honest uncertainty + SDP
\end{itemize}

The idea that any approach can perform well without either having to
know the model or have particularly good data should immediately draw
suspicion. The reader must bear in mind that the strength of our
approach comes not from black-box predictive power from such limited
information, but rather, by providing a more honest expression of
uncertainty outside the observed data without sacrificing the predictive
capacity near the observed data. By coupling this more accurate
description of what is known and unknown to the decision-making under
uncertainty framework provided by stochastic dynamic programming, we are
able to obtain more robust management policies than with common
parametric modeling approaches.

\begin{itemize}
\itemsep1pt\parskip0pt\parsep0pt
\item
  Note on: Why fisheries
\end{itemize}

The economic value and ecological concern have made marine fisheries the
crucible for much of the founding work (Gordon 1954; Reed 1979; May et
al. 1979; Ludwig and Walters 1982) in managing ecosystems under
uncertainty. Global trends (Worm et al. 2006) and controversy (Hilborn
2007; Worm et al. 2009) have made understanding these challenges all the
more pressing.

\begin{itemize}
\itemsep1pt\parskip0pt\parsep0pt
\item
  \emph{Note on comparing models} (via value function rather than by
  ``fit'').
\end{itemize}

\emph{Do we need this?}

The nature of decision-making problems provides a convenient way to
compare models. Rather than compare models in terms of best fit to data
or fret over the appropriate penalty for model complexity, model
performance is defined in the concrete terms of the decision-maker's
objective function, which we will take as given. (Much argument can be
made over the `correct' objective function, e.g.~how to account for the
social value of fish left in the sea vs.~the commercial value of fish
harvested; see Halpern et al. (2013) for further discussion of this
issue. Alternatively, we can always compare model performance across
multiple potential objective functions.) The decision-maker does not
necessarily need a model that provides the best mechanistic
understanding or the best long-term outcome, but rather the one that
best estimates the probabilities of being in different states as a
result of the possible actions.

\subsection{Background on the Gaussian
Process}\label{background-on-the-gaussian-process}

\begin{itemize}
\itemsep1pt\parskip0pt\parsep0pt
\item
  Background on non-parametric modeling.
\end{itemize}

Addressing the difficulty posed by extrapolation without knowing the
true model requires a nonparametric approach to model fitting: one that
does not assume a fixed structure but rather depends on the size of the
data (e.g.~non-parametric regression or a Dirichlet process). This
established terminology is nevertheless unfortunate, as (a) this
approach still involves the estimation of parameters, and (b),
Statisticians use non-parametric to mean both this property (structure
is not fixed by the parameters) and an entirely different (and probably
more familiar) case in which the model does not assume any distribution
(e.g.~non-parametric bootstrap, order statistics). Some literature thus
uses the term semi-parametric, which merely adds ambiguity to the
confusion.

This non-parametric property -- having a structure explicitly dependent
on the data -- is precisely the property that makes this approach
attractive in face of the limited data sampling challenges discussed
above. Having fit a parametric model to some data, the model is
completely described by the values (or posterior distributions) of it's
parameters. The non-parametric model is not captured by its parameter
values or distributions alone. Either the model scales with the
complexity of the data on which it is estimated (e.g.~nonparametric
heirarchical approaches such as the Dirchlet process) or the data points
become themselves part of the model specification, as in the
nonparametric regression used here. we shall see here.

\begin{itemize}
\item
  Definition
\item
  Previous application
\end{itemize}

The use of Gaussian process (GP) regression (or ``kriging'' in the
geospatial literature) to formulate a predictive model is relatively new
in the context of modeling dynamical systems (Kocijan et al. 2005), and
was first introduced in the context ecological modeling and fisheries
management in Munch et al. (2005). An accessible and thorough
introduction to the formulation and use of GPs can be found in Rasmussen
and Williams (2006).

\begin{itemize}
\itemsep1pt\parskip0pt\parsep0pt
\item
  Why it is particularly suited to these two problems
\item
  (Why this is a novel application thereof)
\end{itemize}

The essence of the GP approach can be captured in the following thought
experiment: An exhaustive parametric approach to the challenge of
structural uncertainty might proceed by writing down all possible
functional forms for the underlying dynamical system with all possible
parameter values for each form, and then consider searching over this
huge space to select the most likely model and parameters; or using a
Bayesian approach, assign priors to each of these possible models and
infer the posterior distribution of possible models. The GP approach can
be thought of as a computationally efficient approximation to this
approach. GPs represent a large class of models that can be though of as
capturing or reasonably approximating the set of models in this
collection. By modeling at the level of the process, rather than the
level of parametric equation, we can more concisely capture the possible
behavior of these curves. In place of a parametric model of the
dynamical system, the GP approach postulates a prior distribution of
(n-dimensional) curves that can be though of as approximations to a
range of possible (parametric) models that might describe the data. The
GP allows us to consider probabilities on a large set of possible curves
simultaneously.

The posterior distribution for the hyper-parameters of the Gaussian
process model are estimated by Metropolis-Hastings algorithm, again with
details and code provided in the Appendix. Rasmussen and Williams (2006)
provides an excellent general introduction to Gaussian Processes and
Munch et al. (2005) first discusses their application in the context of
population dynamics models such as fisheries stock-recruitment
relationships.

\section{Approach and Methods}\label{approach-and-methods}

\subsubsection{Summary of approach}\label{summary-of-approach}

\subsubsection{Statement of the optimal control
problem}\label{statement-of-the-optimal-control-problem}

\begin{itemize}
\itemsep1pt\parskip0pt\parsep0pt
\item
  Underlying model
\item
  Available data
\item
  Value function
\end{itemize}

For simplicity we assume profit is simply linear in the realized harvest
(only enforcing the restriction that harvest can not exceed available
stock)

\subsubsection{Parametric models}\label{parametric-models}

\begin{itemize}
\itemsep1pt\parskip0pt\parsep0pt
\item
  Statement of the models
\end{itemize}

We consider three candidate parametric models of the stock-recruitment
dynamics: The Ricker model, the Allen model \href{}{Allen 2005}, the
Myers model. The familiar Ricker model involves two parameters,
corresponding to a growth rate and a carrying capacity, and cannot
support alternative stable state dynamics (though as growth rate
increases it exhibits a periodic attractor that proceeds through
period-doubling into chaos. We will generally focus on dynamics below
the chaotic threshold for the purposes of this analysis.) The Allen
model resembles the Ricker dynamics with an added Allee effect parameter
\href{}{Courchamp}, below which the population cannot persist. The Myers
model also has three parameters and contains an Allee threshold, but has
compensatory rather than over-compensatory density dependence
(resembling a Beverton-Holt curve rather than a Ricker curve at high
densities.)

We assume multiplicative log-normal noise perturbs the growth predicted
by the each of the deterministic model skeletons described above. This
introduces one additional parameter $\sigma$ that must be estimated by
each model.

As we simulate training data from the Allen model (ref section), we will
refer to this as the structurally correct model. The Ricker model is
thus a reasonable approximation of these dynamics far from the Allee
threshold (but lacks threshold dynamics), while the Myers model shares
the essential feature of a threshold but differs in the structure. Thus
we have three potential parametric models of the stock dynamics.

\begin{itemize}
\itemsep1pt\parskip0pt\parsep0pt
\item
  Bayesian inference of parametric models
\end{itemize}

We infer posterior distributions for the parameters of each model in a
Bayesian context using Gibbs sampling (implemented in R ({\textbf{???}})
using jags, ({\textbf{???}})). We choose uninformative uniform priors
for all parameters (See Appendix, Figures S1-S3, and Table S1, and the R
code provided). One-step-ahead predictions of these model fits are shown
in Figure 1.

\begin{itemize}
\itemsep1pt\parskip0pt\parsep0pt
\item
  SDP via parametric models
\end{itemize}

An optimal policy function is then inferred through stochastic dynamic
programming for each model given the posterior distributions of the
parameter estimates. This policy maximizes the expectation of the value
function integrated over the parameter uncertainty. (code implementing
this algorithm provided in the Appendix).

\subsubsection{The Gaussian Process
model}\label{the-gaussian-process-model}

\begin{itemize}
\itemsep1pt\parskip0pt\parsep0pt
\item
  Statement of model
\end{itemize}

\ldots{} more on GP \ldots{} \href{}{Munch 2005}

We also estimate a simple Gaussian Process defined by a radial basis
function kernel of two parameters: $\ell$, which gives the
characteristic length-scale over which correlation between two points
(e.g.~any two points $X_t, X_{t+1}$, and $X_{t+\tau}, X_{t+1+\tau}$) in
state-space decays, and $\sigma$, which gives the scale of the process
noise by which observations $Y_{t+1}$ may differ from their predicted
values $X_{t+1}$ given an observation of the previous state, $X_t$.

\begin{itemize}
\itemsep1pt\parskip0pt\parsep0pt
\item
  Inference of the model
\end{itemize}

Also unlike parametric models, this posterior distribution is still
conditional on the training data. As such, the uncertainty near the
observed data.

We use a Metropolis-Hastings Markov Chain Monte Carlo to infer posterior
distributions of the two parameters of the GP (Figure S4, code in
appendix), under weakly informative Gaussian priors (see parameters in
table S5). As the posterior distributions differ substantially from the
priors (Figure S4), we can be assured that most of the information in
the posterior comes from the data rather than the prior belief.

\begin{itemize}
\itemsep1pt\parskip0pt\parsep0pt
\item
  SDP via the model
\end{itemize}

Though we are unaware of prior application of this type, it is
reasonably straight-forward to adapt the Gaussian Process for Stochastic
Dynamic Programming. Recall that unlike the parametric models the
Gaussian process with fixed parameters already predicts a distribution
of curves rather than a single curve. We must first integrate over this
distribution of curves given a sampling of parameter values drawn from
the posterior distribution of the two GP parameters, before integrating
over the posterior of those parameters themselves.

\section{Results}\label{results}

\begin{figure}[htbp]
\centering
\includegraphics{figure/nonparametric-bayes-Figureb_posteriors.pdf}
\caption{Points show the training data of stock-size over time. Curves
show the posterior step-ahead predictions based on each of the estimated
models.}
\end{figure}

All models fit the observed data rather closely and with relatively
small uncertainty, as illustrated in the posterior predictive curves in
Figure 1. Figure 1 shows the training data of stock sizes observed over
time as points, overlaid with the step-ahead predictions of each
estimated model using the parameters sampled from their posterior
distributions. Each model manages to fit the observed data rather
closely. Compared to the expected value of the true model most estimates
appear to overfit, predicting fluctuations that are actually due purely
to stochasticity in growth rate. Model-choice criteria shown in Table 1
penalize more complex models and show a slight preference for the
simpler Ricker model over the more complicated alternate stable state
models (Allen and Myers). Details on MCMC estimates for each model,
traces, and posterior distributions can be found in the appendix.

\begin{table}[ht]
\begin{center}
\begin{tabular}{rrrr}
  \hline
 & Allen & Ricker & Myers \\ 
  \hline
DIC & 50.14 & 49.45 & 50.61 \\ 
  AIC & -24.60 & -30.07 & -27.19 \\ 
  BIC & -17.85 & -25.00 & -20.44 \\ 
   \hline
\end{tabular}
\end{center}
\end{table}

\begin{figure}[htbp]
\centering
\includegraphics{figure/nonparametric-bayes-statespace_posteriors.pdf}
\caption{Graph of the inferred Gaussian process compared to the true
process and maximum-likelihood estimated process. Graph shows the
expected value for the function $f$ under each model. Two standard
deviations from the estimated Gaussian process covariance with (light
grey) and without (darker grey) measurement error are also shown. The
training data is also shown as black points. The GP is conditioned on
(0,0), shown as a pseudo-data point.}
\end{figure}

The mean inferred state space dynamics of each model relative to the
true model used to generate the data is shown in Figure 2, predicting
the relationship between observed stock size (x-axis) to the stock size
after recruitment the following year. Note that in contrast to the other
models shown, the expected Gaussian process corresponds to a
distribution of curves - as indicated by the gray band - which itself
has a mean shown in black. Parameter uncertainty (not shown) spreads out
the estimates further.\\The observed data from which each model is
estimated is also shown. The observations come from only a limited
region of state space corresponding to unharvested or weakly harvested
system. No observations occur at the theoretical optimum harvest rate or
near the tipping point.

\begin{figure}[htbp]
\centering
\includegraphics{figure/nonparametric-bayes-out_of_sample_predictions.pdf}
\caption{plot of chunk out\_of\_sample\_predictions}
\end{figure}

\begin{figure}[htbp]
\centering
\includegraphics{figure/nonparametric-bayes-Figure2.pdf}
\caption{The steady-state optimal policy (infinite boundary) calculated
under each model. Policies are shown in terms of target escapement,
$S_t$, as under models such as this a constant escapement policy is
expected to be optimal (Reed 1979).}
\end{figure}

Despite the similarities in model fits to the observed data, the
policies inferred under each model differ widely, as shown in Figure
3.\\Policies are shown in terms of target escapement, $S_t$. Under
models such as this a constant escapement policy is expected to be
optimal (Reed 1979), whereby population levels below a certain size $S$
are unharvested, while above that size the harvest strategy aims to
return the population to $S$, resulting in the hockey-stick shaped
policies shown. Only the structurally correct model (Allen model) and
the GP produce policies close to the true optimum policy (where both the
underlying model structure and parameter values are known without
error).

\begin{figure}[htbp]
\centering
\includegraphics{figure/nonparametric-bayes-Figure3.pdf}
\caption{Gaussian process inference outperforms parametric estimates.
Shown are 100 replicate simulations of the stock dynamics (eq 1) under
the policies derived from each of the estimated models, as well as the
policy based on the exact underlying model.}
\end{figure}

The consequences of managing 100 replicate realizations of the simulated
fishery under each of the policies estimated is shown in Figure 4. As
expected from the policy curves, the structurally correct model
under-harvests, leaving the stock to vary around it's un-fished optimum.
The structurally incorrect Ricker model over-harvests the population
passed the tipping point consistently, resulting in the immediate crash
of the stock and thus derives minimal profits.

These results are robust across a range of stochastic realizations,
models, and parameter values.\\The results across this range can most
easily be compared by using the relative differences in net present
value realized by each of the model, as shown in Figure 5. The BNP-SDP
approach most consistently realizes a value close to the optimal
solution, and importantly avoids ever driving the system across the
tipping point, which results in the near-zero value cases in the
parametric models.

\begin{figure}[htbp]
\centering
\includegraphics{figure/nonparametric-bayes-Figure4.pdf}
\caption{Histograms of the realized net present value of the fishery
over a range of simulated data and resulting parameter estimates. For
each data set, the three models are estimated as described above. Values
plotted are the averages of a given policy over 100 replicate
simulations. Details and code provided in the supplement.}
\end{figure}

\section{Discussion}\label{discussion}

\begin{itemize}
\itemsep1pt\parskip0pt\parsep0pt
\item
  All models are ``good fits'' to the originally observed data.
\item
  (Simple model choice immediately leads us astray)
\end{itemize}

\begin{enumerate}
\def\labelenumi{\arabic{enumi}.}
\itemsep1pt\parskip0pt\parsep0pt
\item
  We do not know what the correct models are for ecological systems.
\item
  We have limited data from which to estimate the model -- in
  particular, such models may be misleading in predicting the
  probability of outcomes outside the training data.
\end{enumerate}

These aspects are common to many conservation decision making problems,
which thus merit greater use of non-parametric approaches that can best
take advantage of them.

\subsubsection{Traditional model-choice approaches can be positively
misleading.}\label{traditional-model-choice-approaches-can-be-positively-misleading.}

These results illustrate that model-choice approaches would be
positively misleading -- supporting simpler models that cannot express
tipping point dynamics merely on account of them being similar. As the
data shown comes only from the basin of attraction near the unfished
equilibrium, near which all of the models are approximately linear and
approximately identical.

Model choice approaches trade off model complexity and fit to the data.
When the data come from a limited region of state-space -- as is
necessarily the case whenever there is a potential concern about tipping
point dynamics -- simpler models can fit just as well and will tend to
outperform more complex ones. This approach would be appropriate when
the dynamics can be expected to remain in the region of the training
data; for instance, if we only considered the forecasting accuracy of
the unfished population dynamics under each model.

In contrast, the decision-maker's problem of setting appropriate harvest
levels cannot exclude regions of state-space outside the observed range
when integrating over all possible decisions to find the optimal choice.
Such problems are not constrained to fisheries management but ubiquitous
across ecological decision-making and conservation where the greatest
concerns involve entering previously unobserved regions of state-space
-- whether that is the collapse of a fishery, the spread of an invasive,
or the loss of habitat.

\subsubsection{BNP-SDP expresses larger uncertainty in regions where the
data are
poor}\label{bnp-sdp-expresses-larger-uncertainty-in-regions-where-the-data-are-poor}

The parametric models perform worst when they propose a management
strategy outside the range of the observed data. The non-parametric
Bayesian approach, in contrast, allows a predictive model that expresses
a great deal of uncertainty about the probable dynamics outside the
observed range, while retaining very good predictive accuracy in the
range observed. The management policy dictated by the GP balance this
uncertainty against the immediate value of the harvest, and act to
stabilize the population dynamics in a region of state space in which
the predictions can be reliably reflected by the data.

\subsubsection{BNP-SDP has good predictive accuracy where data are
good}\label{bnp-sdp-has-good-predictive-accuracy-where-data-are-good}

While expressing larger uncertainty outside the observed data, the GP
can also provide a better fit with smaller uncertainty inside the range
of the observed data. This arises from the greater flexibility of the
Gaussian process, which describes a large family of possible curves.
Despite this flexibility, the GP can be described in relatively few
parameters and is thus far less likely to overfit.

\subsection{Future directions}\label{future-directions}

\subsubsection{Higher dimensions}\label{higher-dimensions}

In this simulated example, the underlying dynamics are truly governed by
a simple parametric model, allowing the parametric approaches to be more
accurate. Similarly, because the dynamics are one-dimensional dynamics
and lead to stable nodes (rather than other attractors such as
limit-cycles resulting in oscillations), the training data provides
relatively limited information about the dynamics. For these reasons, we
anticipate that in higher-dimensional examples characteristic of
ecosystem management problems that the machine learning approach will
prove even more valuable.

\subsubsection{Real-time learning}\label{real-time-learning}

In our treatment here we have ignored the possibility of learning during
the management phase, in which the additional observations of the stock
size could potentially improve parameter estimates. While we intend to
address this possibility in future work in the context of these
non-parametric models, we have not addressed it here for pedagogical
reasons. In the context presented here, it is clear that the differences
in performance arise from differences in the uncertainty inherent in the
model formulations, rather than from differing abilities to learn.
Because we consider a threshold system, online learning would not change
this generic feature of a lack of data in a certain range of the state
space which is better captured by the Gaussian process.

\section{Acknowledgments}\label{acknowledgments}

This work was partially supported by the Center for Stock Assessment
Research, a partnership between the University of California Santa Cruz
and the Fisheries Ecology Division, Southwest Fisheries Science Center,
Santa Cruz, CA and by NSF~grant EF-0924195 to MM and NSF grant
DBI-1306697 to CB.

\section{Appendix}\label{appendix}

\subsection{Model definitions and
estimation}\label{model-definitions-and-estimation}

Equation S1: Ricker model.

\[X_{t+1} = Z_t X_t e^{r \left(1 - \frac{S_t}{K} \right) } \]

Figure S1: Ricker model: prior and posterior distributions for parameter
estimates.

\begin{figure}[htbp]
\centering
\includegraphics{figure/nonparametric-bayes-unnamed-chunk-1.pdf}
\caption{plot of chunk unnamed-chunk-1}
\end{figure}

\begin{figure}[htbp]
\centering
\includegraphics{figure/nonparametric-bayes-unnamed-chunk-2.pdf}
\caption{plot of chunk unnamed-chunk-2}
\end{figure}

Table S1: Parameterization of the priors

\begin{table}[ht]
\begin{center}
\begin{tabular}{rlrr}
  \hline
 & parameter & lower\_bound & upper\_bound \\ 
  \hline
1 & r0 & 0.00 & 10.00 \\ 
  2 & K & 0.00 & 40.00 \\ 
  3 & sigma & 0.00 & 100.00 \\ 
   \hline
\end{tabular}
\end{center}
\end{table}

\[ X_{t+1} = Z_t \frac{r S_t^{\theta}}{1 - \frac{S_t^\theta}{K}} \]

Eq S2: Myers model Figure S2: Myers model: Traces, prior and posterior
distributions for parameter estimates.

\begin{figure}[htbp]
\centering
\includegraphics{figure/nonparametric-bayes-unnamed-chunk-4.pdf}
\caption{plot of chunk unnamed-chunk-4}
\end{figure}

\begin{figure}[htbp]
\centering
\includegraphics{figure/nonparametric-bayes-unnamed-chunk-5.pdf}
\caption{plot of chunk unnamed-chunk-5}
\end{figure}

Table S2: Parameterization of the priors

\begin{table}[ht]
\begin{center}
\begin{tabular}{rlrr}
  \hline
 & parameter & lower\_bound & upper\_bound \\ 
  \hline
1 & r0 & 0.00 & 10.00 \\ 
  2 & K & 0.00 & 40.00 \\ 
  3 & theta & 0.00 & 10.00 \\ 
  4 & sigma & 0.00 & 100.00 \\ 
   \hline
\end{tabular}
\end{center}
\end{table}

Eq S3: Allen model

\[f(S_t) = S_t e^{r \left(1 - \frac{S_t}{K}\right)\left(S_t - C\right)} \]

Figure S3: Allen model: prior and posterior distributions for parameter
estimates.

\begin{figure}[htbp]
\centering
\includegraphics{figure/nonparametric-bayes-unnamed-chunk-7.pdf}
\caption{plot of chunk unnamed-chunk-7}
\end{figure}

\begin{figure}[htbp]
\centering
\includegraphics{figure/nonparametric-bayes-unnamed-chunk-8.pdf}
\caption{plot of chunk unnamed-chunk-8}
\end{figure}

Table S3: Parameterization of the priors

\begin{table}[ht]
\begin{center}
\begin{tabular}{rlrr}
  \hline
 & parameter & lower\_bound & upper\_bound \\ 
  \hline
1 & r0 & 0.00 & 10.00 \\ 
  2 & K & 0.00 & 40.00 \\ 
  3 & theta & 0.00 & 10.00 \\ 
  4 & sigma & 0.00 & 100.00 \\ 
   \hline
\end{tabular}
\end{center}
\end{table}

Eq S4: GP model Figure S4: GP model: prior and posterior distributions
for parameter estimates.

\begin{verbatim}
$traces_plot
\end{verbatim}

\begin{figure}[htbp]
\centering
\includegraphics{figure/nonparametric-bayes-unnamed-chunk-101.pdf}
\caption{plot of chunk unnamed-chunk-10}
\end{figure}

\begin{verbatim}

$posteriors_plot
\end{verbatim}

\begin{figure}[htbp]
\centering
\includegraphics{figure/nonparametric-bayes-unnamed-chunk-102.pdf}
\caption{plot of chunk unnamed-chunk-10}
\end{figure}

Table S4: Parameterization of the priors

\subsection{Optimal Control Problem}\label{optimal-control-problem}

We seek the harvest policy $h(x)$ that maximizes:

\[ \max_{h_t} \sum_{t \in 0}^{\infty}  \Pi_t(X_t, h_t) \delta^t  \]

subject to the profit function $\Pi(X_t,h)$, discount rate $\delta$, and
the state equation

\[X_{t+1} = Z_t f(S_t)  \] \[S_t = X_t - h_t \]

Where $Z_t$ is multiplicative noise function with mean 1, representing
stochastic growth. We will consider log-normal noise with shape
parameter $\sigma_g$.

Form this we can write down the Bellman recursion as:

\[V_t(x_t) = \max_h \mathbf{E} \left(\Pi(h_t, x_t) + \delta V_{t+1}( Z_{t+1} f(x_t - h_t)) \right)\]

For simplicity we assume profit is simply linear in the realized harvest
(only enforcing the restriction that harvest can not exceed available
stock), $\Pi(h,x) = min(h,x)$.

\subsubsection{Pseudocode for the Bellman
iteration}\label{pseudocode-for-the-bellman-iteration}

\begin{Shaded}
\begin{Highlighting}[]
 \NormalTok{V1 <-}\StringTok{ }\KeywordTok{sapply}\NormalTok{(}\DecValTok{1}\NormalTok{:}\KeywordTok{length}\NormalTok{(h_grid), function(h)\{}
      \NormalTok{delta *}\StringTok{ }\NormalTok{F[[h]] %*%}\StringTok{ }\NormalTok{V +}\StringTok{  }\KeywordTok{profit}\NormalTok{(x_grid, h_grid[h]) }
    \NormalTok{\})}
    \CommentTok{# find havest, h that gives the maximum value}
    \NormalTok{out <-}\StringTok{ }\KeywordTok{sapply}\NormalTok{(}\DecValTok{1}\NormalTok{:gridsize, function(j)\{}
      \NormalTok{value <-}\StringTok{ }\KeywordTok{max}\NormalTok{(V1[j,], }\DataTypeTok{na.rm =} \NormalTok{T) }\CommentTok{# each col is a diff h, max over these}
      \NormalTok{index <-}\StringTok{ }\KeywordTok{which.max}\NormalTok{(V1[j,])  }\CommentTok{# store index so we can recover h's }
      \KeywordTok{c}\NormalTok{(value, index) }\CommentTok{# returns both profit value & index of optimal h.  }
    \NormalTok{\})}
    \CommentTok{# Sets V[t+1] = max_h V[t] at each possible state value, x}
    \NormalTok{V <-}\StringTok{ }\NormalTok{out[}\DecValTok{1}\NormalTok{,]                        }\CommentTok{# The new value-to-go}
    \NormalTok{D[,OptTime-time}\DecValTok{+1}\NormalTok{] <-}\StringTok{ }\NormalTok{out[}\DecValTok{2}\NormalTok{,]       }\CommentTok{# The index positions}
\end{Highlighting}
\end{Shaded}

\subsubsection{Training data}\label{training-data}

Eacho of our models $f(S_t)$ must be estimated from training data, which
we simulate from the Allen model with parameters $r = $
\texttt{r p{[}1{]}}, $K =$ \texttt{r p{[}2{]}}, $C =$
\texttt{r p{[}3{]}}, and $\sigma_g =$ \texttt{r sigma\_g} for $T=$ 40
timesteps, starting at initial condition $X_0 = $ 5.5. The training data
can be seen in Figure 1.

\begin{center}\rule{3in}{0.4pt}\end{center}

\section{Abstract}\label{abstract-1}

Decision-theoretic methods often rely on simple parametric models of
ecological dynamics to compare the value of a potential sequence of
actions. Unfortunately, such simple models rarely capture the complexity
or uncertainty found in most real ecosystems.

Further, the data on which a model has been parameterized frequently
fails to cover the possible state-space over which management decisions
must operate. Consequently a model do well in the region of state-space
in which it was estimated, but give erroneous confidence to predictions
outside of that region.

This problem is keenly felt in any system where a potential threshold or
tipping point is a concern. Such a tipping point, if it exists at all,
will lay outside the observed range of the observed data.

We demonstrate how nonparametric Bayesian models can provide robust,
solutions to decision making under uncertainty without knowing the
structural form of the true model.

While methods that account for \emph{parametric} uncertainty can be very
successful with the right model, structural uncertainty of not knowing
what model best approximates the dynamics poses considerably greater
difficulty.

\section{Introduction}\label{introduction-1}

\paragraph{Opening}\label{opening}

\paragraph{Models for decision-making under
uncertainty}\label{models-for-decision-making-under-uncertainty}

Decision-theoretic or optimal control tools require a model that can
assign probabilities of future states (e.g.~stock size of a fishery)
given the current state and a proposed action (e.g.~fishing harvest or
effort).\\Management frequently faces a sequential decision-making
problem -- after selecting an action, the decision-maker may receive new
information about the current state and must again choose an appropriate
action -- such as setting the harvest limits each year based on stock
assessments the year prior.

The decision maker typically seeks to determining the course of actions
(also referred to as the policy) that maximizes the expected value of
some objective function such as net present value derived from the
resource over time.\\Though much can be said on how to choose this value
function appropriately (e.g.~see (Halpern et al.
2013)(http://doi.org/10.1073/pnas.1217689110 ``Achieving the triple
bottom line in the face of inherent trade-offs among social equity,
economic return, and conservation.'')) we will assume this is given.
(Nor is this approach necessarily constrained to maximizing the
expectated value of such a function - the decision-theoretic framework
can be adapted to alternatives such as minimizing the maximum cost or
damage that might be incurred; see Polasky et al. (2011)).

In representing future states with probabilities and maximizing
expectations, this approach provides a natural framework for handling
uncertainty.

The value function typically depends on the action or policy taken, as
well as the state of the system, in each interval of time. The state of
the system, in turn, is usually described by a dynamical model.

(B. K. Williams 2001; Athanassoglou and Xepapadeas 2012).

While simple mechanistic models can nevertheless provide important
insights into long-term outcomes, such approaches are not well-suited
for use in forecasting outcomes of potential management options.
Non-parametric approaches offer a more flexible alternative that can
both more accurately reflect the data available while also representing
greater uncertainty in areas (of state-space) where data is lacking.

We demonstrate how a Gaussian Process model of stock recruitment can
lead to nearly optimal management through stochastic dynamic
programming, comperable to knowing the correct structural equation for
the underlying simulation. Meanwhile, parametric models that do not
match the underlying dynamics can perform very poorly, even though they
fit the data as well as the true model.\\Ecological research and
management strategy should pay closer attention to the opportunities and
challenges nonparametric modeling can offer.

\section{Approach and Methods}\label{approach-and-methods-1}

\subsection{The optimal control problem in fisheries
management}\label{the-optimal-control-problem-in-fisheries-management}

We focus on the problem in which a manager must set the harvest level
for a marine fishery each year to maximize the net present value of the
resource, given an estimated stock size from the year before.

To permit comparisons against a theoretical optimum we will consider
data on the stock dynamics simulated from a simple parametric model in
which recruitment of the fish stock $X_{t+1}$ in the following year is a
stochastic process governed by a function $f$ of the current stock
$X_t$, selected harvest policy $h_t$, and noise process $Z$,

\[X_{t+1} = Z_t f(X_t, h_t) \]

Given parameters for the function $f$ and probability distribution $Z$,
along with a given economic model determining the price/profit
$\Pi(X_t, h_t)$ realized in a given year given a choice of harvest $h_t$
and observed stock $X_t$. This problem can be solved exactly for
discretized values of stock $X$ and policy $h$ using stochastic dynamic
programming (SDP) (Mangel and Clark 1988). Problems of this sort
underpin much marine fisheries management today.

A crux of this approach is correctly specifying the functional form of
$f$, along with its parameters. The standard approach uses one of a
handful of common parametric models representing the stock-recruitment
relationship, usually after estimating the model parameters from any
available existing data. Uncertainty in the parameter estimates can be
estimated and integrated over to determine the optimal policy under
under uncertainty (Mangel and Clark 1988; Schapaugh and Tyre 2013).
Uncertainty in the model structure itself can only be addressed in this
approach by hypothesizing alternative model structures, and then
performing some model choice or model averaging (B. K. Williams 2001;
Athanassoglou and Xepapadeas 2012).

\subsection{Underlying Model}\label{underlying-model}

To illustrate the value of the non-parametric Bayesian approach to
management, we focus on example of a system containing such a tipping
point whose dynamics can still be described by a simple, one-dimensional
parametric model.\\We will focus on a simple parametric model for a
single species (derived from fist principles by Allen et al. 2005) as
our underlying ``reality''.

\[X_{t+1} = Z_t f(S_t)  \] \[S_t = X_t - h_t \]
\[f(S_t) = S_t e^{r \left(1 - \frac{S_t}{K}\right)\left(S_t - C\right)} \]

Where $Z_t$ is multiplicative noise function with mean 1, representing
stochastic growth. We will consider log-normal noise with shape
parameter $\sigma_g$. We start with an example in which the parameters
are $r =2$, $K=8$, $C=5$ and $\sigma_g = 0.1$.

As a low-dimensional system completely described by three parameters,
this scenario should if anything be favorable to a parametric-based
approach. This model contains an Allee effect, or tipping point, below
which the population is not self-sustaining and shrinks to zero
(Courchamp, Berec, and Gascoigne 2008).

\paragraph{Simulated training data}\label{simulated-training-data}

We generate initial observational data under the model described in Eq 1
for $T_{\textrm{obs}}=40$ time steps, under a given arbitrary sequence
of harvest intensities, $h_t$. We consider the case in which most of the
data comes from a limited region of state space (e.g.~near a stable
equilibrium), leaving us without observations of the population dynamics
at very low levels which would be useful in discrimating between
recruitment curves {[}@{]} or demonstrating the existence of a tipping
point (Scheffer et al. 2001).\\Using data simulated from a specified
model rather than empirical data permits the comparison against the true
underlying dynamics, setting a bar for the optimal performance possible.

\subsubsection{Parametric Models}\label{parametric-models-1}

We consider three candidate parametric models for the stock-recruitment
function, which we refer to by the first authors of the publications in
which they were first proposed.

We generate the data with a four-parameter model that contains a tipping
point, as discussed above (equation 1), (an Allee effect, see
({\textbf{???}})) below which the stock decreases to zero,

\[ X_{t+1} = Z_t S_t e^{r \left(1 - \frac{S_t}{K}\right)\left(\frac{S_t - \theta}{K}\right)} \]

\[ S_t = X_t - h_t \]

The parameter $C$ reflects the location of the tipping point, $K$ the
carrying capacity of the stock, and $r$ the base recruitment rate. $S_t$
represents the stock size after a harvest $h_t$ has been implemented.
$Z_t$ represents a log-normal random variable of log-mean zero and
log-standard deviation parameter $\sigma$.

We consider two alternative candidate models: the Ricker
({\textbf{???}}) stock-recruitment curve,

\[X_{t+1} = Z_t X_t e^{r \left(1 - \frac{S_t}{K} \right) } \]

and an alternative four-parameter model adapted from ({\textbf{???}}),

\[ X_{t+1} = Z_t \frac{r S_t^{\theta}}{1 - \frac{S_t^\theta}{K}} \]

which contains a tipping point for $\theta > 2$ and becomes a
Beverton-Holt model at $\theta = 1$.

\subsubsection{Bayesian Inference of Parametric
models}\label{bayesian-inference-of-parametric-models}

Given the sample data, we infer posterior distributions for each of the
three models listed above using a Markov Chain Monte Carlo Gibbs Sampler
(jags, see appendix for implementation details and code) given uniform
priors. We run six chains for $10^6$ steps each and then assess
convergence by Gelman-Rubin criterion and inspection of the traces, see
appendix.

\subsubsection{The Non-parametric Bayesian alternative for
stock-recruitment
curves}\label{the-non-parametric-bayesian-alternative-for-stock-recruitment-curves}

\subsubsection{SDP via GP}\label{sdp-via-gp}

Once the posterior Gaussian process (GP) has been estimated (e.g.~see
Munch et al. 2005), it is necessary to adapt it in place of the
parametric equation for the stochastic dynamic programming (SDP)
solution (see Mangel and Clark 1988 for a detailed description of
parametric SDP methods) to the optimal policy. The essence of the idea
is straight forward -- we will use the estimated GP in place of the
parametric growth function to determine the stochastic transition matrix
on which the SDP calculations are based. The SDP is solved in a
discretized state space -- both the continuously valued population
densities $X$ and harvest quotas $h$ are first mapped to a bounded,
discrete grid. (For simplicity we will consider a uniform grid, though
for either parametric or GP-based SDP it is often advantageous to use a
non-uniform discretization such as a basis function representation,
e.g.~see (Deisenroth, Rasmussen, and Peters 2009)).

The SDP approach then computes a transition matrix, $\mathbf{F}$. We
demonstrate that calculation is just as straight forward based on the GP
as it is in the classical context using the parametric model. The
${i,j}$ of the transition matrix $F$ entry gives the probability of
transitioning into state $x_i$ given that the system is in state $x_j$
in the previous time-step. To generate the transition matrix based on
the posterior GP, we need only the expected values at each grid point
and the corresponding variances (the diagonal of the covariance matrix),
as shown in Figure 1. Given the mean of the GP posterior at each
grid-point as the vector $E$ and variance at that point as vector $V$,
the probability of transitioning from state $x_i$ to state $x_j$ is

\[\mathcal{N}\left(x_j | \mu = E_i, \sigma = \sqrt{V_i}\right)\]

where $\mathcal{N}$ is the Normal density at $x_j$ with mean $\mu$ and
variance $\sigma^2$. Strictly speaking, the transition probability
should be calculated by integrating the normal density over the bin of
width $\Delta$ centered at $x_j$. For a sufficiently fine grid that
$f(x_j) \approx f(x_j + \Delta)$, it is sufficient to calculate the
density at $x_j$ and then row-normalize the transition matrix. The
process can then be repeated for each possible discrete value of our
control variable, (harvest $h$).

\textbf{Pseudocode for the determining the transition matrix from the
GP}

\begin{Shaded}
\begin{Highlighting}[]
\NormalTok{for(h in h_grid)}
  \NormalTok{F_h =}\StringTok{ }\NormalTok{for(x_j in grid)}
          \NormalTok{for(i in }\DecValTok{1}\NormalTok{:N) }
            \KeywordTok{dnorm}\NormalTok{(x_j, mu[i]-h, V[i])}
\end{Highlighting}
\end{Shaded}

Using the discrete transition matrix we may write down the Bellman
recursion defining the stochastic dynamic programming iteration:

\begin{equation}
V_t(x_t) = \max_h \mathbf{E} \left( h_t + \delta V_{t+1}( Z_{t+1} f(x_t - h_t)) \right)
\end{equation}

where $V(x_t)$ is the value of being at state $x$ at time $t$, $h$ is
control (harvest level) chosen. Numerically, the maximization is
accomplished as follows. Consider the set of possible control values to
be the discrete values corresponding the grid of stock sizes. Then for
each $h_t$ there is a corresponding transition matrix $\mathbf{F}_h$
determined as described above but with mean $\mu = x_j - h_t$. Let
$\vec{V_t}$ be the vector whose $i$th element corresponds to the value
of having stock $x_i$ at time $t$. Then let $\Pi_h$ be the vector whose
$i$th element indicates the profit from harvesting at intensity $h_t$
given a population $x_i$ (e.g. $\max(x_i, h_t)$ since one cannot harvest
more fish then the current population size). Then the Bellman recursion
can be given in matrix form as

\[V_{t} = \max_h \left( \Pi_{h_{t}} + \delta \mathbf{F}_h V_{t+1} \right)\]

where the sum is element by element and the expectation is computed by
the matrix multiplication $\mathbf{F} V_{t+1}$.

\subsubsection{Pseudocode for the Bellman
iteration}\label{pseudocode-for-the-bellman-iteration-1}

\begin{Shaded}
\begin{Highlighting}[]
 \NormalTok{V1 <-}\StringTok{ }\KeywordTok{sapply}\NormalTok{(}\DecValTok{1}\NormalTok{:}\KeywordTok{length}\NormalTok{(h_grid), function(h)\{}
      \NormalTok{delta *}\StringTok{ }\NormalTok{F[[h]] %*%}\StringTok{ }\NormalTok{V +}\StringTok{  }\KeywordTok{profit}\NormalTok{(x_grid, h_grid[h]) }
    \NormalTok{\})}
    \CommentTok{# find havest, h that gives the maximum value}
    \NormalTok{out <-}\StringTok{ }\KeywordTok{sapply}\NormalTok{(}\DecValTok{1}\NormalTok{:gridsize, function(j)\{}
      \NormalTok{value <-}\StringTok{ }\KeywordTok{max}\NormalTok{(V1[j,], }\DataTypeTok{na.rm =} \NormalTok{T) }\CommentTok{# each col is a diff h, max over these}
      \NormalTok{index <-}\StringTok{ }\KeywordTok{which.max}\NormalTok{(V1[j,])  }\CommentTok{# store index so we can recover h's }
      \KeywordTok{c}\NormalTok{(value, index) }\CommentTok{# returns both profit value & index of optimal h.  }
    \NormalTok{\})}
    \CommentTok{# Sets V[t+1] = max_h V[t] at each possible state value, x}
    \NormalTok{V <-}\StringTok{ }\NormalTok{out[}\DecValTok{1}\NormalTok{,]                        }\CommentTok{# The new value-to-go}
    \NormalTok{D[,OptTime-time}\DecValTok{+1}\NormalTok{] <-}\StringTok{ }\NormalTok{out[}\DecValTok{2}\NormalTok{,]       }\CommentTok{# The index positions}
\end{Highlighting}
\end{Shaded}

This completes the algorithm adapting the GP to the sequential
decision-making problem through SDP, which has not been previously
demonstrated.\\We further provide an R package implementation as
described in the supplemental materials.

\subsubsection{Estimating parametric
models}\label{estimating-parametric-models}

We estimate posterior distributions for two parametric models: one using
the structurally correct model as given in Eq (1), which we refer to as
the ``Parametric Bayes'' model, and another using the familiar Ricker
model, using a Gibbs sampler as described (with source code) in the
appendix). In addition we estimate the parameters of the structurally
correct model by maximum likelihood.

\section{Results}\label{results-1}

\section{Discussion}\label{discussion-1}

\paragraph{Big picture: Linking GP to
SDP}\label{big-picture-linking-gp-to-sdp}

\emph{rambling}

Non-parametric Bayesian methods have received far too little attention
in ecological modeling efforts that are aimed at improved conservation
planning and decision making support. Such approaches may be
particularly useful when the available data is restricted to a limited
area of state-space, which can lead parametric models to underestimate
the uncertainty in dynamics at population levels (states) which have not
been observed. One reason for the relative absence of nonparametric
approaches in the natural resource management context may be the lack of
existing approaches for adapting the non-parametric Bayesian models
previously proposed (Munch et al. 2005) to a decision-theoretic
framework. Adapting a non-parametric approach requires modification of
existing methods for decision theory. We have illustrated how this might
be done for a classic stochastic dynamic programming problem, opening
the door for substantial further research into how these applications
might be improved.

\begin{center}\rule{3in}{0.4pt}\end{center}

Code to replicate the analysis, along with complete log of this research
can be found at:
\href{https://github.com/cboettig/nonparametric-bayes/}{https://github.com/cboettig/nonparametric-bayes}

\subsection{Markov Chain Monte Carlo
Analysis}\label{markov-chain-monte-carlo-analysis}

Allen, Linda J. S., Jesse F. Fagan, Göran Högnäs, and Henrik Fagerholm.
2005. ``Population Extinction in Discrete-time Stochastic Population
Models with an Allee Effect.'' \emph{Journal of Difference Equations and
Applications} 11 (4-5) (apr): 273--293.
doi:10.1080/10236190412331335373.
\url{http://www.tandfonline.com/doi/abs/10.1080/10236190412331335373}.

Athanassoglou, Stergios, and Anastasios Xepapadeas. 2012. ``Pollution
Control with Uncertain Stock Dynamics: When, and How, to Be
Precautious.'' \emph{Journal of Environmental Economics and Management}
63 (3) (may): 304--320. doi:10.1016/j.jeem.2011.11.001.
\url{http://linkinghub.elsevier.com/retrieve/pii/S0095069611001409}.

Bestelmeyer, Brandon T., Michael C. Duniway, Darren K. James, Laura M.
Burkett, and Kris M. Havstad. 2012. ``A Test of Critical Thresholds and
Their Indicators in a Desertification-prone Ecosystem: More Resilience
Than We Thought.'' Edited by Katharine Suding. \emph{Ecology Letters}
(dec). doi:10.1111/ele.12045.
\url{http://doi.wiley.com/10.1111/ele.12045}.

Brozović, Nicholas, and Wolfram Schlenker. 2011. ``Optimal Management of
an Ecosystem with an Unknown Threshold.'' \emph{Ecological Economics}
(jan): 1--14. doi:10.1016/j.ecolecon.2010.10.001.
\url{http://linkinghub.elsevier.com/retrieve/pii/S0921800910004167}.

Clark, Colin W., and Geoffrey P. Kirkwood. 1986. ``On Uncertain
Renewable Resource Stocks: Optimal Harvest Policies and the Value of
Stock Surveys.'' \emph{Journal of Environmental Economics and
Management} 13 (3) (sep): 235--244. doi:10.1016/0095-0696(86)90024-0.
\url{http://linkinghub.elsevier.com/retrieve/pii/0095069686900240}.

Courchamp, Franck, Ludek Berec, and Joanna Gascoigne. 2008. \emph{Allee
Effects in Ecology and Conservation}. Oxford University Press, USA.
\url{http://www.amazon.com/Effects-Ecology-Conservation-Franck-Courchamp/dp/0198570309}.

Cressie, Noel, Catherine a Calder, James S. Clark, Jay M. Ver Hoef, and
Christopher K. Wikle. 2009. ``Accounting for Uncertainty in Ecological
Analysis: the Strengths and Limitations of Hierarchical Statistical
Modeling.'' \emph{Ecological Applications} 19 (3) (apr): 553--70.
doi:10.1890/07-0744.1.
\url{http://www.ncbi.nlm.nih.gov/pubmed/19425416}.

Deisenroth, Marc Peter, Carl Edward Rasmussen, and Jan Peters. 2009.
``Gaussian Process Dynamic Programming.'' \emph{Neurocomputing} 72 (7-9)
(mar): 1508--1524. doi:10.1016/j.neucom.2008.12.019.
\url{http://linkinghub.elsevier.com/retrieve/pii/S0925231209000162}.

Gordon, H. S. 1954. ``The Economic Theory of a Common-property Resource:
the Fishery.'' \emph{The Journal of Political Economy} 62 (2): 124--142.
\url{http://www.jstor.org/stable/10.2307/1825571}.

Halpern, Benjamin S., Carissa J. Klein, Christopher J. Brown, Maria
Beger, Hedley S. Grantham, Sangeeta Mangubhai, Mary Ruckelshaus, et al.
2013. ``Achieving the Triple Bottom Line in the Face of Inherent
Trade-offs Among Social Equity, Economic Return, and Conservation.''
\emph{Proceedings of the National Academy of Sciences of the United
States of America} 110 (15) (apr): 6229--34.
doi:10.1073/pnas.1217689110.
\href{http://www.ncbi.nlm.nih.gov/pubmed/23530207 http://www.pubmedcentral.nih.gov/articlerender.fcgi?artid=3625307/\&tool=pmcentrez/\&rendertype=abstract}{http://www.ncbi.nlm.nih.gov/pubmed/23530207
http://www.pubmedcentral.nih.gov/articlerender.fcgi?artid=3625307\textbackslash{}\&tool=pmcentrez\textbackslash{}\&rendertype=abstract}.

Hilborn, Ray. 2007. ``Reinterpreting the State of Fisheries and Their
Management.'' \emph{Ecosystems} 10 (8) (oct): 1362--1369.
doi:10.1007/s10021-007-9100-5.
\url{http://www.springerlink.com/index/10.1007/s10021-007-9100-5}.

Hughes, Terry P., Cristina Linares, Vasilis Dakos, Ingrid a van de
Leemput, and Egbert H. van Nes. 2013. ``Living Dangerously on Borrowed
Time During Slow, Unrecognized Regime Shifts.'' \emph{Trends in Ecology
\& Evolution} 28 (3) (mar): 149--55. doi:10.1016/j.tree.2012.08.022.
\url{http://www.ncbi.nlm.nih.gov/pubmed/22995893}.

Kocijan, Juš, Agathe Girard, Bla\textbackslash{}vz Banko, and Roderick
Murray-Smith. 2005. ``Dynamic Systems Identification with Gaussian
Processes.'' \emph{Mathematical and Computer Modelling of Dynamical
Systems} 11 (4) (dec): 411--424. doi:10.1080/13873950500068567.
\url{http://www.tandfonline.com/doi/abs/10.1080/13873950500068567}.

Ludwig, Donald, and Carl J. Walters. 1982. ``Optimal Harvesting with
Imprecise Parameter Estimates.'' \emph{Ecological Modelling} 14 (3-4)
(jan): 273--292. doi:10.1016/0304-3800(82)90023-0.
\url{http://linkinghub.elsevier.com/retrieve/pii/0304380082900230}.

Mangel, Marc, and Colin W. Clark. 1988. \emph{Dynamic Modeling in
Behavioral Ecology}. Edited by John Krebs and Tim Clutton-Brock.
Princeton: Princeton University Press.

May, Robert M., John R. Beddington, Colin W. Clark, Sidney J. Holt, and
R. M. Laws. 1979. ``Management of Multispecies Fisheries.''
\emph{Science (New York, N.Y.)} 205 (4403) (jul): 267--77.
doi:10.1126/science.205.4403.267.
\url{http://www.ncbi.nlm.nih.gov/pubmed/17747032}.

Munch, Stephan B., Athanasios Kottas, and Marc Mangel. 2005. ``Bayesian
Nonparametric Analysis of Stock–recruitment Relationships.''
\emph{Canadian Journal of Fisheries and Aquatic Sciences} 62 (8) (aug):
1808--1821. doi:10.1139/f05-073.
\url{http://www.nrcresearchpress.com/doi/abs/10.1139/f05-073}.

Munch, Stephan B., Melissa L. Snover, George M. Watters, and Marc
Mangel. 2005. ``A Unified Treatment of Top-down and Bottom-up Control of
Reproduction in Populations.'' \emph{Ecology Letters} 8 (7) (may):
691--695. doi:10.1111/j.1461-0248.2005.00766.x.
\url{http://doi.wiley.com/10.1111/j.1461-0248.2005.00766.x}.

Polasky, Stephen, Stephen R. Carpenter, Carl Folke, and Bonnie Keeler.
2011. ``Decision-making Under Great Uncertainty: Environmental
Management in an Era of Global Change.'' \emph{Trends in Ecology \&
Evolution} (may): 1--7. doi:10.1016/j.tree.2011.04.007.
\url{http://www.ncbi.nlm.nih.gov/pubmed/21616553}.

Rasmussen, Carl Edward, and C. K. I. Williams. 2006. \emph{Gaussian
Processes for Machine Learning}. Edited by Thomas Dietterich. Boston:
MIT Press,. \url{www.GaussianProcess.org/gpml}.

Reed, William J. 1979. ``Optimal Escapement Levels in Stochastic and
Deterministic Harvesting Models.'' \emph{Journal of Environmental
Economics and Management} 6 (4) (dec): 350--363.
doi:10.1016/0095-0696(79)90014-7.
\href{http://www.sciencedirect.com/science/article/pii/0095069679900147 http://linkinghub.elsevier.com/retrieve/pii/0095069679900147}{http://www.sciencedirect.com/science/article/pii/0095069679900147
http://linkinghub.elsevier.com/retrieve/pii/0095069679900147}.

Schapaugh, Adam W., and Andrew J. Tyre. 2013. ``Accounting for
Parametric Uncertainty in Markov Decision Processes.'' \emph{Ecological
Modelling} 254 (apr): 15--21. doi:10.1016/j.ecolmodel.2013.01.003.
\url{http://linkinghub.elsevier.com/retrieve/pii/S0304380013000306}.

Scheffer, Marten, Stephen R. Carpenter, J. A. Foley, C. Folke, and B.
Walker. 2001. ``Catastrophic Shifts in Ecosystems.'' \emph{Nature} 413
(6856) (oct): 591--6. doi:10.1038/35098000.
\url{http://www.ncbi.nlm.nih.gov/pubmed/11595939}.

Sethi, Gautam, Christopher Costello, Anthony Fisher, Michael Hanemann,
and Larry Karp. 2005. ``Fishery Management Under Multiple Uncertainty.''
\emph{Journal of Environmental Economics and Management} 50 (2) (sep):
300--318. doi:10.1016/j.jeem.2004.11.005.
\url{http://linkinghub.elsevier.com/retrieve/pii/S0095069605000057}.

Weitzman, Martin L. 2013. ``A Precautionary Tale of Uncertain Tail
Fattening.'' \emph{Environmental and Resource Economics} 55 (2) (mar):
159--173. doi:10.1007/s10640-013-9646-y.
\url{http://link.springer.com/10.1007/s10640-013-9646-y}.

Williams, Byron K. 2001. ``Uncertainty , Learning , and the Optimal
Management of Wildlife.'' \emph{Environmental and Ecological Statistics}
8: 269--288. doi:10.1023/A:1011395725123.

Worm, Boris, Edward B. Barbier, Nicola Beaumont, J. Emmett Duffy, Carl
Folke, Benjamin S. Halpern, Jeremy B. C. Jackson, et al. 2006. ``Impacts
of Biodiversity Loss on Ocean Ecosystem Services.'' \emph{Science (New
York, N.Y.)} 314 (5800) (nov): 787--90. doi:10.1126/science.1132294.
\url{http://www.ncbi.nlm.nih.gov/pubmed/17082450}.

Worm, Boris, Ray Hilborn, Julia K. Baum, Trevor A. Branch, Jeremy S.
Collie, Christopher Costello, Michael J. Fogarty, et al. 2009.
``Rebuilding Global Fisheries.'' \emph{Science (New York, N.Y.)} 325
(5940) (jul): 578--85. doi:10.1126/science.1173146.
\url{http://www.ncbi.nlm.nih.gov/pubmed/19644114}.

\end{document}


