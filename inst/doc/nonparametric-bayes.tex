\documentclass[author-year, review]{elsarticle} %review=doublespace preprint=single 5p=2 column
%%% Begin My package additions %%%%%%%%%%%%%%%%%%%
\usepackage[hyphens]{url}
\usepackage{lineno} % add 
%\linenumbers % turns line numbering on 
\bibliographystyle{elsarticle-harv}
\biboptions{sort&compress} % For natbib
\usepackage{graphicx}
\usepackage{booktabs} % book-quality tables
%% Redefines the elsarticle footer
\makeatletter
\def\ps@pprintTitle{%
 \let\@oddhead\@empty
 \let\@evenhead\@empty
 \def\@oddfoot{\it \hfill\today}%
 \let\@evenfoot\@oddfoot}
\makeatother

% A modified page layout
\textwidth 6.75in
\oddsidemargin -0.15in
\evensidemargin -0.15in
\textheight 9in
\topmargin -0.5in
%%%%%%%%%%%%%%%% end my additions to header



\usepackage[T1]{fontenc}
\usepackage{lmodern}
\usepackage{amssymb,amsmath}
\usepackage{ifxetex,ifluatex}
\usepackage{fixltx2e} % provides \textsubscript
% use upquote if available, for straight quotes in verbatim environments
\IfFileExists{upquote.sty}{\usepackage{upquote}}{}
\ifnum 0\ifxetex 1\fi\ifluatex 1\fi=0 % if pdftex
  \usepackage[utf8]{inputenc}
\else % if luatex or xelatex
  \usepackage{fontspec}
  \ifxetex
    \usepackage{xltxtra,xunicode}
  \fi
  \defaultfontfeatures{Mapping=tex-text,Scale=MatchLowercase}
  \newcommand{\euro}{€}
\fi
% use microtype if available
\IfFileExists{microtype.sty}{\usepackage{microtype}}{}
\ifxetex
  \usepackage[setpagesize=false, % page size defined by xetex
              unicode=false, % unicode breaks when used with xetex
              xetex]{hyperref}
\else
  \usepackage[unicode=true]{hyperref}
\fi
\hypersetup{breaklinks=true,
            bookmarks=true,
            pdfauthor={},
            pdftitle={Avoiding tipping points in the management of ecological systems: a non-parametric Bayesian approach},
            colorlinks=true,
            urlcolor=blue,
            linkcolor=magenta,
            pdfborder={0 0 0}}
\urlstyle{same}  % don't use monospace font for urls
\setlength{\parindent}{0pt}
\setlength{\parskip}{6pt plus 2pt minus 1pt}
\setlength{\emergencystretch}{3em}  % prevent overfull lines
\setcounter{secnumdepth}{0}
% Pandoc toggle for numbering sections (defaults to be off)
\setcounter{secnumdepth}{0}
% Pandoc header



\begin{document}
\begin{frontmatter}

  \title{Avoiding tipping points in the management of ecological systems: a
non-parametric Bayesian approach}
    \author[cstar]{Carl Boettiger\corref{c1}}
   \ead{cboettig@gmail.com} 
   \cortext[c1]{Corresponding author}
    \author[cstar]{Marc Mangel}
  
  
    \author[noaa]{Stephan Munch}
  
  
      \address[cstar]{Center for Stock Assessment Research, Department of Applied Math and
Statistics, University of California, Mail Stop SOE-2, Santa Cruz, CA
95064, USA}    
    \address[noaa]{Southwest Fisheries Science Center, National Oceanic and Atmospheric
Administration, 110 Shaffer Road, Santa Cruz, CA 95060, USA}    
  
  \begin{abstract}
  Model uncertainty and limited data coverage are fundamental challenges
  to robust ecosystem management. These challenges are acutely highlighted
  by concerns that many ecological systems may contain tipping points.
  Before a collapse, we do not know where the tipping points lie, if the
  exist at all. Hence, we know neither a complete model of the system
  dynamics nor do we have access to data in some large region of
  state-space where such a tipping point might exist. These two sources of
  uncertainty frustrate state-of-the-art parametric approaches to decision
  theory and optimal control. I will illustrate how a non-parametric
  approach using a Gaussian Process prior provides a more flexible
  representation of this inherent uncertainty. Consequently, we can adapt
  the Gaussian Process prior to a stochastic dynamic programming framework
  in order to make robust management predictions under both model and
  uncertainty and limited data.
  \end{abstract}
   \begin{keyword} Bayesian \sep Structural Uncertainty \sep Nonparametric \sep Optimal Control \sep Decision Theory \sep Gaussian Processes \sep Fisheries Management \sep \end{keyword}
 \end{frontmatter}


\section{Introduction}\label{introduction}

Decision making under uncertainty is a ubiquitous challenge of natural
resource management and conservation. Ecological dynamics are frequently
complex and difficult to measure, making uncertainty in our
understanding a prediction a persistent challenge to effective
management. Decision-theoretic approaches provide a framework to
determine the best sequence of actions in face of uncertainty, but only
when that uncertainty can be meaningfully quantified (Fischer et al.
2009). The sudden collapse of fisheries and other ecosystems has
increasingly emphasized the difficulties of formulating even
qualitatively correct models of the underlying processes.

We develop a novel approach to address these concerns in the context of
fisheries; though the underlying challenges and methods are germane to
many other conservation and resource management problems. The economic
value and ecological concern have made marine fisheries the crucible for
much of the founding work (H. S. Gordon 1954; Reed 1979; {\textbf{???}};
Ludwig and Walters 1982) in managing ecosystems under uncertainty.
Global trends (Worm et al. 2006) and controversy (Hilborn 2007;
{\textbf{???}}) have made understanding these challenges all the more
pressing.

Uncertainty enters the decision-making process at many levels: intrinsic
stochasticity in biological processes, measurements, and implementation
of policy (\emph{e.g.} Reed 1979; C. W. Clark and Kirkwood 1986;
Roughgarden and Smith 1996; Sethi et al. 2005), parametric uncertainty
(\emph{e.g.} Ludwig and Walters 1982; {\textbf{???}}; McAllister 1998;
{\textbf{???}}), and model or structural uncertainty (\emph{e.g.} B. K.
Williams 2001; Cressie et al. 2009; Athanassoglou and Xepapadeas 2012).
Of these, structural uncertainty incorporates the least a priori
knowledge or assumptions and is generally the hardest to quantify.
Typical approaches assume a weak notion of model uncertainty in which
the correct model (or reasonable approximation) of the dynamics must be
identified from among a handful of alternative models. Here we consider
an approach that addresses uncertainty at each of these levels without
assuming the dynamics follow a particular (i.e.~parametric) structure.

\emph{Cut the next three paragraphs, since they are covered more
concisely in the above paragraph?}

\subsubsection{Process, measurement, and implementation
error}\label{process-measurement-and-implementation-error}

Resource management and conservation planning seek to determine the
optimal set of feasible actions to maximize the value of some objectives
(e.g Halpern et al. (2013)). Process error, measurement error,
implementation error (Reed 1979). These sources of stochasticity in turn
mean that model parameters can only be estimated approximately,
requiring parametric uncertainty also be considered (Ludwig and Walters
1982).

\subsubsection{Parametric uncertainty}\label{parametric-uncertainty}

As the parameter values for these models must be estimated from limited
data, there will always be some uncertainty associated with these
values. This uncertainty further compounds the intrinsic variability
introduced by demographic or environmental noise. The degree of
uncertainty in the parameter values can be inferred from the data and
reflected in the estimates of the transition probabilities (Ludwig and
Walters 1982; Mangel and Clark 1988; {\textbf{???}}; {\textbf{???}}).

\subsubsection{Structural (model)
uncertainty}\label{structural-model-uncertainty}

Estimates of parameter uncertainty are only as good as the parametric
models themselves. Often we do not understand the system dynamics well
enough to know if a model provides a good approximation over the
relevant range of states and timescales (criteria that we loosely refer
to as defining the ``right'' or ``true'' model.) So called structural or
model uncertainty is a more difficult problem than parametric
uncertainty. Typical solutions involve either model choice, model
averaging, or introducing yet greater model complexity of which others
may be special cases (model averaging being one such way to construct
such a model) (B. K. Williams 2001; Athanassoglou and Xepapadeas 2012;
Cressie et al. 2009). Even setting aside other computational and
statistical concerns (e.g. (Cressie et al. 2009)), these approaches do
not address our second concern - representing uncertainty outside the
observed data range.

Model uncertainty is particularly insidious when model predictions must
be made outside of the range of data on which the model was estimated.
This extrapolation uncertainty is felt most keenly in decision-theoretic
(or optimal control) applications, since (a) exploring the potential
action space typically involves considering actions that may move the
system outside the range of observed behavior, and (b)
decision-theoretic alogrithms rely not only on reasonable estimates of
the expected outcomes, but depend on the weights given to all possible
outcomes (\emph{e.g.} Weitzman 2013). If we are observing the
fluctuations of a given fish stock over many years under a fixed
harvesting pressure, we might develop and test a model that could
reasonably predict the frequency of a deviation of a given size, even
when such a deviation has not been previously observed. Yet such
predictions are far less reliable when extrapolated to a harvest
pressure that has not yet been observed. Thus, model uncertainty can be
particularly challenging in the management and decision-making context.

This difficult position of having neither the true model nor data that
covers the full range of possible states is unfortunately the rule more
than the exception. The potential concern of tipping points in
ecological dynamics (Scheffer et al. 2001; Polasky et al. 2011) reflects
these concerns -- as either knowledge of the true model or more complete
sampling of the state space would make it easy to identify if a tipping
point existed. If we do not know but cannot rule out such a possibility,
then we face decision-making under this dual challege of model
uncertainty and incomplete data coverage.

These dual concerns pose a substantial challenge to existing
decision-theoretic approaches (Brozović and Schlenker 2011). Because
intervention is often too late after a tipping point has been crossed
(but see Hughes et al. (2013)), management is most often concerned with
avoiding potentially catastrophic tipping points before any data is
available at or following a transition that would more clearly reveal
these regime shift dynamics (e.g. Bestelmeyer et al. 2012).

Here we illustrate how a stochastic dynamic programming (SDP) algorithm
(Mangel and Clark 1988; Marescot et al. 2013) can be driven by the
predictions from a Bayesian non-parametric (BNP) approach (Munch,
Kottas, and Mangel 2005). This provides two distinct advantages compared
with contemporary approaches. First, using a BNP sidesteps the need for
an accurate model-based description of the system dynamics. Second, the
BNP can better reflect uncertainty that arises when extrapolating a
model outside of the data on which it was fit. We illustrate that when
the correct model is not known, this latter feature is crucial to
providing a robust decision-theoretic approach in face of substantial
structural uncertainty.

This paper represents the first time the SDP decision-making framework
has been used without an a priori model of the underlying dynamics
through the use of the BNP approach. In contrast to parametric models
which can only reflect uncertainty in parameter estimates, the BNP
approach provides a more state-space dependent representation of
uncertainty. This permits a much greater uncertainty far from the
observed data than near the observed data. These features allow the
BNP-SDP approach to find robust management solutions in face of limited
data and without knowledge of the correct model structure.

The idea that any approach can perform well without either having to
know the model or have particularly good data should immediately draw
suspicion. The reader must bear in mind that the strength of our
approach comes not from black-box predictive power from such limited
information, but rather, by providing a more honest expression of
uncertainty outside the observed data without sacrificing the predictive
capacity near the observed data. By coupling this more accurate
description of what is known and unknown to the decision-making under
uncertainty framework provided by stochastic dynamic programming, we are
able to obtain more robust management policies than with common
parametric modeling approaches.

The nature of decision-making problems provides a convenient way to
compare models. Rather than compare models in terms of best fit to data
or fret over the appropriate penalty for model complexity, model
performance is defined in the concrete terms of the decision-maker's
objective function, which we will take as given. (Much argument can be
made over the `correct' objective function, e.g.~how to account for the
social value of fish left in the sea vs.~the commercial value of fish
harvested; see Halpern et al. (2013) for further discussion of this
issue. Alternatively, we can always compare model performance across
multiple potential objective functions.) The decision-maker does not
necessarily need a model that provides the best mechanistic
understanding or the best long-term outcome, but rather the one that
best estimates the probabilities of being in different states as a
result of the possible actions.

\subsection{Background on the Gaussian
Process}\label{background-on-the-gaussian-process}

Addressing the difficulty posed by extrapolation without knowing the
true model requires a nonparametric approach to model fitting: one that
does not assume a fixed structure but rather depends on the size of the
data (e.g.~non-parametric regression or a Dirichlet process). This
established terminology is nevertheless unfortunate, as (a) this
approach still involves the estimation of parameters, and (b),
Statisticians use non-parametric to mean both this property (structure
is not fixed by the parameters) and an entirely different (and probably
more familiar) case in which the model does not assume any distribution
(e.g.~non-parametric bootstrap, order statistics). Some literature thus
uses the term semi-parametric, which merely adds ambiguity to the
confusion.

This non-parametric property -- having a structure explicitly dependent
on the data -- is precisely the property that makes this approach
attractive in face of the limited data sampling challenges discussed
above. Having fit a parametric model to some data, the model is
completely described by the values (or posterior distributions) of it's
parameters. The non-parametric model is not captured by its parameter
values or distributions alone. Either the model scales with the
complexity of the data on which it is estimated (e.g.~nonparametric
hierarchical approaches such as the Dirchlet process) or the data points
become themselves part of the model specification, as in the
nonparametric regression used here.

The use of Gaussian process (GP) regression (or ``kriging'' in the
geospatial literature) to formulate a predictive model is relatively new
in the context of modeling dynamical systems ({\textbf{???}}), and was
first introduced in the context ecological modeling and fisheries
management in Munch et al. (2005). An accessible and thorough
introduction to the formulation and use of GPs can be found in Rasmussen
and Williams (2006).

The posterior distribution for the hyper-parameters of the Gaussian
process model are estimated by Metropolis-Hastings algorithm, again with
details and code provided in the Appendix. Rasmussen and Williams (2006)
provides an excellent general introduction to Gaussian Processes and
Munch, Kottas, and Mangel (2005) first discusses their application in
the context of population dynamics models such as fisheries
stock-recruitment relationships.

\section{Approach and Methods}\label{approach-and-methods}

\subsubsection{Statement of the optimal control
problem}\label{statement-of-the-optimal-control-problem}

To illustrate the application of the BNP-SDP approach and compare to the
predictions of the alternative parametric models we focus on the
classical problem of selecting the appropriate harvest level given an
observation of the stock size in the previous year (Reed 1979; Ludwig
and Walters 1982; Mangel and Clark 1988). Given this observation and the
model (together with the parameter uncertainty) of the stock recruitment
process, the manager seeks to maximize the value of the fishery over a
fixed time interval of 50 years at a discount rate of 0.01. The value
function (profits) at time $t$ depends on the true stock size $x_t$ and
the chosen harvest level $h_t$. For simplicity we assume profit is
simply proportional in the realized harvest (only enforcing the
restriction that harvest can not exceed available stock).

\subsubsection{Parametric models}\label{parametric-models}

We consider three candidate parametric models of the stock-recruitment
dynamics: The Ricker model, the Allen model (Allen and Tanner 2005), the
Myers model (Myers et al. 1995). The familiar Ricker model involves two
parameters, corresponding to a growth rate and a carrying capacity, and
cannot support alternative stable state dynamics (though as growth rate
increases it exhibits a periodic attractor that proceeds through
period-doubling into chaos. We will generally focus on dynamics below
the chaotic threshold for the purposes of this analysis.) The Allen
model resembles the Ricker dynamics with an added Allee effect parameter
({\textbf{???}}), below which the population cannot persist. The Myers
model also has three parameters and contains an Allee threshold, but has
compensatory rather than over-compensatory density dependence
(resembling a Beverton-Holt curve rather than a Ricker curve at high
densities.)\\We assume multiplicative log-normal noise perturbs the
growth predicted by the each of the deterministic model skeletons
described above. This introduces one additional parameter $\sigma$ that
must be estimated by each model.

As we simulate training data from the Allen model, we will refer to this
as the structurally correct model. The Ricker model is thus a reasonable
approximation of these dynamics far from the Allee threshold (but lacks
threshold dynamics), while the Myers model shares the essential feature
of a threshold but differs in the structure. Thus we have three
potential parametric models of the stock dynamics.

We introduce parametric uncertainty by first estimating each of the
candidate models from data on unexploited stock dynamics following some
perturbation (non-equilibrium initial condition) over several time
steps. This training data could be generated in several different ways
(such as known variable exploitation rates, etc.), as long as it
reflects the dynamics in some limited region of state space without
impacting the problem. We consider a period of 40 years of training
data: long enough that the estimates are not dependent on the particular
realization, while longer times are not likely to provide substantial
improvement (i.e.~the results are not sensitive to this interval). Each
of the models (described below) is fit to the same training data, as
shown in Figure 1.

We infer posterior distributions for the parameters of each model in a
Bayesian context using Gibbs sampling (implemented in R (Team 2013)
using jags, (Su and Yajima 2013)). We choose uninformative uniform
priors for all parameters (See Appendix, Figures and tables, and the R
code provided). One-step-ahead predictions of these model fits are shown
in Figure 1. While alternative approaches to the estimation of the
posteriors (such as integrating out the rate parameter $r$ analytically
and then performing a grid search over the remaining parameter space),
the approach of using a standard Gibbs sampler routine is both more
general and representitive of common practice in estimating posteriors
for such models. Each sampling is tested for Gelman-Rubin convergence
and results are robust to longer runs.

An optimal policy function is then inferred through stochastic dynamic
programming for each model given the posterior distributions of the
parameter estimates. This policy maximizes the expectation of the value
function integrated over the parameter uncertainty. (code implementing
this algorithm provided in the Appendix).

\subsubsection{The Gaussian Process
model}\label{the-gaussian-process-model}

We also estimate a simple Gaussian Process defined by a radial basis
function kernel of two parameters: $\ell$, which gives the
characteristic length-scale over which correlation between two points in
state-space decays, and $\sigma$, which gives the scale of the process
noise by which observations $Y_{t+1}$ may differ from their predicted
values $X_{t+1}$ given an observation of the previous state, $X_t$.
Munch, Kottas, and Mangel (2005) gives an accessible introduction to the
use of Gaussian Processes in providing a Bayesian nonparametric
description of the stock-recruitment relationship.

We use a Metropolis-Hastings Markov Chain Monte Carlo to infer posterior
distributions of the two parameters of the GP (Figure S13, code in
appendix), under weakly informative Gaussian priors (see parameters in
table S5). As the posterior distributions differ substantially from the
priors (Figure S13), we can be assured that most of the information in
the posterior comes from the data rather than the prior belief.

Though we are unaware of prior application of this type, it is
reasonably straight-forward to adapt the Gaussian Process for Stochastic
Dynamic Programming. Recall that unlike the parametric models the
Gaussian process with fixed parameters already predicts a distribution
of curves rather than a single curve. We must first integrate over this
distribution of curves given a sampling of parameter values drawn from
the posterior distribution of the two GP parameters, before integrating
over the posterior of those parameters themselves.

\section{Results}\label{results}

Allen, Deborah, and Kimberly Tanner. 2005. ``Infusing Active Learning
into the Large-enrollment Biology Class: Seven Strategies, from the
Simple to Complex.'' \emph{Cell Biology Education} 4 (4) (jan): 262--8.
\url{http://www.ncbi.nlm.nih.gov/pubmed/16344858}.

Athanassoglou, Stergios, and Anastasios Xepapadeas. 2012. ``Pollution
Control with Uncertain Stock Dynamics: When, and How, to Be
Precautious.'' \emph{Journal of Environmental Economics and Management}
63 (3) (may): 304--320. doi:10.1016/j.jeem.2011.11.001.
\url{http://linkinghub.elsevier.com/retrieve/pii/S0095069611001409}.

Bestelmeyer, Brandon T., Michael C. Duniway, Darren K. James, Laura M.
Burkett, and Kris M. Havstad. 2012. ``A Test of Critical Thresholds and
Their Indicators in a Desertification-prone Ecosystem: More Resilience
Than We Thought.'' Edited by Katharine Suding. \emph{Ecology Letters}
(dec). doi:10.1111/ele.12045.
\url{http://doi.wiley.com/10.1111/ele.12045}.

Brozović, Nicholas, and Wolfram Schlenker. 2011. ``Optimal Management of
an Ecosystem with an Unknown Threshold.'' \emph{Ecological Economics}
(jan): 1--14. doi:10.1016/j.ecolecon.2010.10.001.
\url{http://linkinghub.elsevier.com/retrieve/pii/S0921800910004167}.

Clark, Colin W., and Geoffrey P. Kirkwood. 1986. ``On Uncertain
Renewable Resource Stocks: Optimal Harvest Policies and the Value of
Stock Surveys.'' \emph{Journal of Environmental Economics and
Management} 13 (3) (sep): 235--244. doi:10.1016/0095-0696(86)90024-0.
\url{http://linkinghub.elsevier.com/retrieve/pii/0095069686900240}.

Cressie, Noel, Catherine a Calder, James S. Clark, Jay M. Ver Hoef, and
Christopher K. Wikle. 2009. ``Accounting for Uncertainty in Ecological
Analysis: the Strengths and Limitations of Hierarchical Statistical
Modeling.'' \emph{Ecological Applications : a Publication of the
Ecological Society of America} 19 (3) (apr): 553--70.
\url{http://www.ncbi.nlm.nih.gov/pubmed/19425416}.

Fischer, Joern, Garry D. Peterson, Toby a Gardner, Line J. Gordon, Ioan
Fazey, Thomas Elmqvist, Adam Felton, Carl Folke, and Stephen Dovers.
2009. ``Integrating Resilience Thinking and Optimisation for
Conservation.'' \emph{Trends in Ecology \& Evolution} 24 (10) (oct):
549--54. doi:10.1016/j.tree.2009.03.020.
\url{http://www.ncbi.nlm.nih.gov/pubmed/19665820}.

Gordon, H. S. 1954. ``The Economic Theory of a Common-property Resource:
the Fishery.'' \emph{The Journal of Political Economy} 62 (2): 124--142.
\url{http://www.jstor.org/stable/10.2307/1825571}.

Halpern, Benjamin S., Carissa J. Klein, Christopher J. Brown, Maria
Beger, Hedley S. Grantham, Sangeeta Mangubhai, Mary Ruckelshaus, et al.
2013. ``Achieving the Triple Bottom Line in the Face of Inherent
Trade-offs Among Social Equity, Economic Return, and Conservation.''
\emph{Proceedings of the National Academy of Sciences of the United
States of America} 110 (15) (apr): 6229--34.
doi:10.1073/pnas.1217689110.
\href{http://www.ncbi.nlm.nih.gov/pubmed/23530207 http://www.pubmedcentral.nih.gov/articlerender.fcgi?artid=3625307/\&tool=pmcentrez/\&rendertype=abstract}{http://www.ncbi.nlm.nih.gov/pubmed/23530207
http://www.pubmedcentral.nih.gov/articlerender.fcgi?artid=3625307\textbackslash{}\&tool=pmcentrez\textbackslash{}\&rendertype=abstract}.

Hilborn, Ray. 2007. ``Reinterpreting the State of Fisheries and Their
Management.'' \emph{Ecosystems} 10 (8) (oct): 1362--1369.
doi:10.1007/s10021-007-9100-5.
\url{http://www.springerlink.com/index/10.1007/s10021-007-9100-5}.

Hughes, Terry P., Cristina Linares, Vasilis Dakos, Ingrid a van de
Leemput, and Egbert H. van Nes. 2013. ``Living Dangerously on Borrowed
Time During Slow, Unrecognized Regime Shifts.'' \emph{Trends in Ecology
\& Evolution} 28 (3) (mar): 149--55. doi:10.1016/j.tree.2012.08.022.
\url{http://www.ncbi.nlm.nih.gov/pubmed/22995893}.

Ludwig, Donald, and Carl J. Walters. 1982. ``Optimal Harvesting with
Imprecise Parameter Estimates.'' \emph{Ecological Modelling} 14 (3-4)
(jan): 273--292. doi:10.1016/0304-3800(82)90023-0.
\url{http://linkinghub.elsevier.com/retrieve/pii/0304380082900230}.

Mangel, Marc, and Colin W. Clark. 1988. \emph{Dynamic Modeling in
Behavioral Ecology}. Edited by John Krebs and Tim Clutton-Brock.
Princeton: Princeton University Press.

Marescot, Lucile, Guillaume Chapron, Iadine Chadès, Paul L. Fackler,
Christophe Duchamp, Eric Marboutin, and Olivier Gimenez. 2013. ``Complex
Decisions Made Simple: a Primer on Stochastic Dynamic Programming.''
\emph{Methods in Ecology and Evolution} (jun).
doi:10.1111/2041-210X.12082.
\url{http://doi.wiley.com/10.1111/2041-210X.12082}.

McAllister, M. K. 1998. ``Bayesian Stock Assessment: a Review and
Example Application Using the Logistic Model.'' \emph{ICES Journal of
Marine Science} 55 (6) (dec): 1031--1060. doi:10.1006/jmsc.1998.0425.
\href{http://icesjms.oxfordjournals.org/content/55/6/1031.short http://icesjms.oxfordjournals.org/cgi/doi/10.1006/jmsc.1998.0425}{http://icesjms.oxfordjournals.org/content/55/6/1031.short
http://icesjms.oxfordjournals.org/cgi/doi/10.1006/jmsc.1998.0425}.

Munch, Stephan B., Athanasios Kottas, and Marc Mangel. 2005. ``Bayesian
Nonparametric Analysis of Stock–recruitment Relationships.''
\emph{Canadian Journal of Fisheries and Aquatic Sciences} 62 (8) (aug):
1808--1821. doi:10.1139/f05-073.
\url{http://www.nrcresearchpress.com/doi/abs/10.1139/f05-073}.

Munch, Stephan B., Melissa L. Snover, George M. Watters, and Marc
Mangel. 2005. ``A Unified Treatment of Top-down and Bottom-up Control of
Reproduction in Populations.'' \emph{Ecology Letters} 8 (7) (may):
691--695. doi:10.1111/j.1461-0248.2005.00766.x.
\url{http://doi.wiley.com/10.1111/j.1461-0248.2005.00766.x}.

Myers, R. a, N. J. Barrowman, J. a Hutchings, and a a Rosenberg. 1995.
``Population Dynamics of Exploited Fish Stocks at Low Population
Levels.'' \emph{Science (New York, N.Y.)} 269 (5227) (aug): 1106--8.
doi:10.1126/science.269.5227.1106.
\url{http://www.ncbi.nlm.nih.gov/pubmed/17755535}.

Polasky, Stephen, Stephen R. Carpenter, Carl Folke, and Bonnie Keeler.
2011. ``Decision-making Under Great Uncertainty: Environmental
Management in an Era of Global Change.'' \emph{Trends in Ecology \&
Evolution} (may): 1--7. doi:10.1016/j.tree.2011.04.007.
\url{http://www.ncbi.nlm.nih.gov/pubmed/21616553}.

Rasmussen, Carl Edward, and C. K. I. Williams. 2006. \emph{Gaussian
Processes for Machine Learning}. Edited by Thomas Dietterich. Boston:
MIT Press,. \url{www.GaussianProcess.org/gpml}.

Reed, William J. 1979. ``Optimal Escapement Levels in Stochastic and
Deterministic Harvesting Models.'' \emph{Journal of Environmental
Economics and Management} 6 (4) (dec): 350--363.
doi:10.1016/0095-0696(79)90014-7.
\href{http://www.sciencedirect.com/science/article/pii/0095069679900147 http://linkinghub.elsevier.com/retrieve/pii/0095069679900147}{http://www.sciencedirect.com/science/article/pii/0095069679900147
http://linkinghub.elsevier.com/retrieve/pii/0095069679900147}.

Roughgarden, Joan E., and F. Smith. 1996. ``Why Fisheries Collapse and
What to Do About It.'' \emph{Proceedings of the National Academy of
Sciences of the United States of America} 93 (10): 5078.
\url{http://www.pnas.org/content/93/10/5078.short}.

Scheffer, Marten, Stephen R. Carpenter, J. A. Foley, C. Folke, and B.
Walker. 2001. ``Catastrophic Shifts in Ecosystems.'' \emph{Nature} 413
(6856) (oct): 591--6. doi:10.1038/35098000.
\url{http://www.ncbi.nlm.nih.gov/pubmed/11595939}.

Sethi, Gautam, Christopher Costello, Anthony Fisher, Michael Hanemann,
and Larry Karp. 2005. ``Fishery Management Under Multiple Uncertainty.''
\emph{Journal of Environmental Economics and Management} 50 (2) (sep):
300--318. doi:10.1016/j.jeem.2004.11.005.
\url{http://linkinghub.elsevier.com/retrieve/pii/S0095069605000057}.

Su, Yu-Sung, and Masanao Yajima. 2013. \emph{R2jags: A Package for
Running Jags from R}. \url{http://cran.r-project.org/package=R2jags}.

Team, R. Core. 2013. \emph{R: A Language and Environment for Statistical
Computing}. Vienna, Austria: R Foundation for Statistical Computing.
\url{http://www.r-project.org/}.

Weitzman, Martin L. 2013. ``A Precautionary Tale of Uncertain Tail
Fattening.'' \emph{Environmental and Resource Economics} 55 (2) (mar):
159--173. doi:10.1007/s10640-013-9646-y.
\url{http://link.springer.com/10.1007/s10640-013-9646-y}.

Williams, Byron K. 2001. ``Uncertainty , Learning , and the Optimal
Management of Wildlife.'' \emph{Environmental and Ecological Statistics}
8: 269--288. doi:10.1023/A:1011395725123.

Worm, Boris, Edward B. Barbier, Nicola Beaumont, J. Emmett Duffy, Carl
Folke, Benjamin S. Halpern, Jeremy B. C. Jackson, et al. 2006. ``Impacts
of Biodiversity Loss on Ocean Ecosystem Services.'' \emph{Science (New
York, N.Y.)} 314 (5800) (nov): 787--90. doi:10.1126/science.1132294.
\url{http://www.ncbi.nlm.nih.gov/pubmed/17082450}.

\end{document}


