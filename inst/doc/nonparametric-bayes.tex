\documentclass[author-year, review]{elsarticle} %review=doublespace preprint=single 5p=2 column
\usepackage{amsmath, amsfonts, amssymb}  % extended mathematics
% My package additions
\usepackage[hyphens]{url}
\usepackage{lineno} % add 
%\linenumbers % turns line numbering on 
\bibliographystyle{elsarticle-harv}
\biboptions{sort&compress} % For natbib
\usepackage{graphicx}
\usepackage{booktabs} % book-quality tables

%% Redefines the elsarticle footer
\makeatletter
\def\ps@pprintTitle{%
 \let\@oddhead\@empty
 \let\@evenhead\@empty
 \def\@oddfoot{\it \hfill\today}%
 \let\@evenfoot\@oddfoot}
\makeatother

% A modified page layout
\textwidth 6.75in
\oddsidemargin -0.15in
\evensidemargin -0.15in
\textheight 9in
\topmargin -0.5in


\usepackage{microtype}
\usepackage{fancyhdr}
\pagestyle{fancy}
\pagenumbering{arabic}

\usepackage{listings}
\lstnewenvironment{code}{\lstset{language=Haskell,basicstyle=\small\ttfamily}}{}


\setlength{\parindent}{0pt}
\setlength{\parskip}{6pt plus 2pt minus 1pt}


%%% Syntax Highlighting for code  %%%
%%% Adapted from knitr book %%% 
\usepackage{fancyvrb}
\DefineVerbatimEnvironment{Highlighting}{Verbatim}{commandchars=\\\{\}}
% Add ',fontsize=\small' for more characters per line
\newenvironment{Shaded}{}{}
\newcommand{\KeywordTok}[1]{\textcolor[rgb]{0.00,0.44,0.13}{\textbf{{#1}}}}
\newcommand{\DataTypeTok}[1]{\textcolor[rgb]{0.56,0.13,0.00}{{#1}}}
\newcommand{\DecValTok}[1]{\textcolor[rgb]{0.25,0.63,0.44}{{#1}}}
\newcommand{\BaseNTok}[1]{\textcolor[rgb]{0.25,0.63,0.44}{{#1}}}
\newcommand{\FloatTok}[1]{\textcolor[rgb]{0.25,0.63,0.44}{{#1}}}
\newcommand{\CharTok}[1]{\textcolor[rgb]{0.25,0.44,0.63}{{#1}}}
\newcommand{\StringTok}[1]{\textcolor[rgb]{0.25,0.44,0.63}{{#1}}}
\newcommand{\CommentTok}[1]{\textcolor[rgb]{0.38,0.63,0.69}{\textit{{#1}}}}
\newcommand{\OtherTok}[1]{\textcolor[rgb]{0.00,0.44,0.13}{{#1}}}
\newcommand{\AlertTok}[1]{\textcolor[rgb]{1.00,0.00,0.00}{\textbf{{#1}}}}
\newcommand{\FunctionTok}[1]{\textcolor[rgb]{0.02,0.16,0.49}{{#1}}}
\newcommand{\RegionMarkerTok}[1]{{#1}}
\newcommand{\ErrorTok}[1]{\textcolor[rgb]{1.00,0.00,0.00}{\textbf{{#1}}}}
\newcommand{\NormalTok}[1]{{#1}}
\usepackage{enumerate}
\usepackage{ctable}
\usepackage{float}

% This is needed because raggedright in table elements redefines \\:
\newcommand{\PreserveBackslash}[1]{\let\temp=\\#1\let\\=\temp}
\let\PBS=\PreserveBackslash
\usepackage[normalem]{ulem}
\newcommand{\textsubscr}[1]{\ensuremath{_{\scriptsize\textrm{#1}}}}

% Configure hyperlinks package
\usepackage[breaklinks=true,linktocpage,pdftitle={},pdfauthor={},colorlinks]{hyperref}
\hypersetup{breaklinks=true, pdfborder={0 0 0}}

% Pandoc toggle for numbering sections (defaults to be off)
\setcounter{secnumdepth}{0}


\VerbatimFootnotes % allows verbatim text in footnotes

% Pandoc header



\begin{document}
\begin{frontmatter}
  \title{}
  \author[cstar]{Carl Boettiger\corref{cor1}}
  \author[noaa]{Stephan B. Munch}
  \author[cstar]{Marc Mangel}
  \ead{cboettig@ucsc.edu}
  \cortext[cor1]{Corresponding author, cboettig@ucsc.edu}
  \address[cstar]{Center for Stock Assessment Research, Department of Applied Math and Statistics, University of California, Mail Stop SOE-2, Santa Cruz, CA 95064, USA}
  \address[noaa]{Southwest Fisheries Science Center, National Oceanic and Atmospheric Administration, 110 Shaffer Road, Santa Cruz, CA 95060, USA}
 \end{frontmatter}


\section{Non-parametric approaches to optimal policy are more robust}

Carl Boettiger, Marc Mangel, Steve Munch

\section{Abstract}

\section{Introduction}

The problem of structural uncertainty in managing ecological systems.

\begin{itemize}
\item
  Most management recommendations from the ecological literature /
  management policies based on (motivated by) parametric models.
  \emph{Preference to pitch towards policy or theoretical literature?}
\item
  Background discussion on the importance/success of parametric
  modeling; (e.g. Levins (1966), etc., up through Geritz and Kisdi
  (2011))
\item
  Background on the concerns of structural uncertainty -- we don't have
  the right models. (Also: measurement uncertainty, parameter
  uncertainty, unobserved states, boundary conditions, etc.)
\item
  Hierarchical Bayesian approach has provided a natural way to address
  these from a statistical standpoint. Successes and challenges from the
  parametric route: e.g. Cressie et al. (2009).
\item
  Among pitfalls of these approaches: particularly difficult to apply in
  management context. Optimization-based (Decision-theoretic) approaches
  to ecological management need to be precisely parameterized about
  everything, even the uncertainty itself (Polasky et al. 2011)
  (discussion of Brozović and Schlenker (2011) SDP example in context of
  threshold system would perhaps be useful as well).
\item
  Management goals / decision-theoretic approaches need accurate
  prediction over relevant (short?) timescales more than accurate (but
  incomplete or noisy estimated) mechanisms.
\item
  Nonparametric (machine-learning?) approaches may offer the benefit of
  the hierarchical Bayesian approach without the practical and
  computational limitations of their parametric kin. Non-parametric
  models are flexible enough to take maximum advantage of the data
  available, while being appropriately ambiguous about the dynamics of a
  system in regions of parameter space that have been poorly or never
  sampled.
\item
  Nonparametric approaches are beginning to appear more frequently in
  ecological and conservation literature (Species distribution
  models/maxent), including the Gaussian process based approach used
  here (Munch et al. 2005). However, such approaches have yet to be
  applied to the decision-theoretic framework that could guide
  management decisions, where we expect them to excel for several
  reasons (1) the ability to make accurate forecasts by more closely
  approximating the underlying process where data is available (2)
  remaining appropriately ambiguous where data is not available (3)
  remaining computationally simple enough to avoid some pitfalls common
  to hierarchical parametric approaches.
\end{itemize}

\section{Approach and Methods}

\subsubsection{Discussion of state equation}

Concerns over the potential for tipping points in ecological dynamics
(Scheffer et al. 2001) highlight the dangers of uncertainty in
ecological management and pose a substantial challenge to existing
decision-theoretic approaches (Brozović and Schlenker 2011). To compare
the performance of nonparametric and parametric approaches in an example
that is easy to conceptualize, we will focus on a simple parametric
model for a single species (derived from fist principles by Allen et al.
2005) as our underlying ``reality''.

\begin{align}
X_{t+1} &= Z_t f(S_t) \\
S_t &= X_t - h_t \\
f(S_t) &= e^{r \left(1 - \frac{S_t}{K}\right)\left(S_t - C\right)}
\end{align}

As a low-dimensional system completely described by three parameters,
this scenario should if anything be favorable to a parametric-based
approach. This model contains an Allee effect, or tipping point, below
which the population is not self-sustaining and shrinks to zero
(Courchamp, Berec, and Gascoigne 2008).

\subsubsection{Brief discussion on choice of model parameters, nuisance
parameters}

Where $Z_t$ is multiplicative noise function with mean 1, representing
stochastic growth. We will consider log-normal noise with shape
parameter $\sigma_g$. We start with an example in which the parameters
are $r =$ \texttt{2}, $K =$ \texttt{10}, $C =$ \texttt{5}, and
$\sigma_g =$ \texttt{0.05}.

\subsubsection{Discussion of training data}

Both parametric and nonparametric approaches will require some training
data on which to base their model of the process. We generate the
training data under the model described in Eq 1 for \texttt{35} time
steps, under a known but not necessarily optimal sequence of harvest
intensities, $h_t$. For simplicity we imagine a fishery that started
from zero harvest pressure and has been gradually increasing the
harvest. (Motivation, alternatives, stationarity, examples without a
stable node (limit-cycle models), examples based on observations near a
stable node alone, and why that isn't impossible).

\subsubsection{Discussion of maximum likelihood estimated models}

We estimate two parametric models from the data using a maximum
likelihood approach. The first model is structurally identical to the
true model (Eq 1), differing only in that it's parameters are estimated
from the observed data rather than given. The alternative model is the
Ricker model, which is structurally similar and commonly assumed

(MLE models will assume the noise is log-normal, which it is in the
simulation).

Which estimates a Ricker model with $r =$ \texttt{1.8501}, $K =$
\texttt{9.8091}, and the Allen Allele model with $r =$ \texttt{2.8079},
$K =$ \texttt{11.8235} and $C =$ \texttt{7.2159}.

\subsubsection{(Brief) Discussion of GP inference}

\begin{itemize}
\item
  The use of Gaussian processes for inference in dynamical systems
  {[}introduced by @Kocijan2005{]}
\item
  Gaussian processes in ecological literature (Munch et al. 2005)
\item
  Our methodology (e.g.~following Munch et al. (2005)).
\end{itemize}

\subsubsection{Discussion of the dynamic programming solution}

\emph{(More thorough, but general-audience targeted. Technical details
and code provided in appendices).}

The fishery management problem over an infinite time horizon can be
stated as:

\begin{align}
& \max_{ \{h_t\} \geq 0 } \mathbf{E} \lbrace \sum_0^\infty \delta^t \Pi(h_t) \rbrace \\
& \mathrm{s.t.}  \\
 & X_t = Z_t f\left(S_{t-1}\right) \\
 & S_t = X_t - h_t \\
 & X_t  \geq 0 
\end{align}

Where $\mathbf{E}$ is the expectation operator, $\delta$ the discount
rate, $\Pi(h_t)$ the profit expected from a harvest of $h_t$, and other
terms as in Eq. (1). For simplicity, we have assumed that profits depend
only on the chosen harvest; simplifying further we will usually consider
profits to be proportional to harvest, $\Pi(h_t) = h_t$.

Once the posterior Gaussian process (GP) has been estimated (e.g.~see
Munch et al. 2005), it is necessary to adapt it in place of the
parametric equation for the stochastic dynamic programming (SDP)
solution (see Mangel and Clark 1988 for a detailed description of
parametric SDP methods) to the optimal policy. The essense of the idea
is straight forward -- we will use the estimated GP in place of the
parametric growth function to determine the stochastic transition matrix
on which the SDP calculations are based.

The posterior Gaussian process is completely defined by the expected
value and covariance matrix at a defined set of training points. For
simplicty we will consider a these points to fall on a discrete, uniform
grid $x$ of \texttt{101} points from \texttt{0} to \texttt{15} (1.5
times the positive equilibrium $K$). Again to keep things simple we will
use this same grid discritization for the parametric approach. Other
options for choosing the grid points, including collocation methods and
functional basis expansion (or even using Guassian processes in place of
the discrete optimization; an entirely different context in which GP can
be used in SDP, see {[}@Deisenroth2009{]}) could also be considered.

The transition matrix $\mathbf{F}$ is thus an \texttt{101} by
\texttt{101} matrix for which the ${i,j}$ entry gives the probability of
transitioning into state $x_i$ given that the system is in state $x_j$
in the previous timestep. To generate the transition matrix based on the
posterior GP, we need only the expected values at each grid point and
the corresponding variances (the diagonal of the covariance matrix), as
shown in Figure 1. Given the mean at each gridpoint as the length
\texttt{101} vector $E$ and variance $V$, the probability of
transitioning from state $x_i$ to state $x_j$ is simply
$\mathcal{N}\left(x_j | \mu = E_i, \sigma = \sqrt{V_i}\right)$, where
$\mathcal{N}$ is the Normal density at $x_j$ with mean $\mu$ and
variance $\sigma^2$. Strictly speaking, the transition probability
should be calculated by integrating the normal density over the bin of
width $\Delta$ centered at $x_j$. For a sufficiently fine grid that
$f(x_j) \approx f(x_j + \Delta)$, it is sufficient to calculate the
density at $x_j$ and then row-normalize the transition matrix.

\subsection{Pseudocode for the determining the transtion matrix from the
GP}

\begin{Shaded}
\begin{Highlighting}[]
\NormalTok{for(h in h_grid)}
  \NormalTok{F_h = for(x_j in grid)}
          \NormalTok{for(i in }\DecValTok{1}\NormalTok{:N) }
            \KeywordTok{dnorm}\NormalTok{(x_j, mu[i]-h, V[i])}
\end{Highlighting}
\end{Shaded}

A transition matrix for each of the parametric models $f$ is calculated
using the log-normal density with mean $f(x_i)$ and log-variance as
estimated by maximum likelihood. From the discrete transition matrix we
may write down the Bellman recursion defining the the stochastic dynamic
programming iteration:

\begin{equation}
V_t(x_t) = \max_h \mathbf{E} \left( h_t + \delta V_{t+1}( Z_{t+1} f(x_t - h_t)) \right)
\end{equation}

where $V(x_t)$ is the value of being at state $x$ at time $t$, $h$ is
control (harvest level) chosen. Numerically, the maximization is
accomplished as follows. Consider the set of possible control values to
be the discrete \texttt{101} values corresponding the the grid of stock
sizes. Then for each $h_t$ there is a corresponding transition matrix
$\mathbf{F}_h$ determined as described above but with mean
$\mu = x_j - h_t$. Let $\vec{V_t}$ be the vector whose $i$th element
corresponds to the value of having stock $x_i$ at time $t$. Then let
$\Pi_h$ be the vector whose $i$th element indicates the profit from
harvesting at intensity $h_t$ given a population $x_i$ (e.g.
$\max(x_i, h_t)$ since one cannot harvest more fish then the current
population size). Then the Bellman recursion can be given in matrix form
as

\[V_{t} = \max_h \left( \Pi_{h_{t}} + \delta \mathbf{F}_h V_{t+1} \right)\]

where the sum is element by element and the expectation is computed by
the matrix multiplication $\mathbf{F} V_{t+1}$.

\subsection{Pseudocode for the Bellman iteration}

\begin{Shaded}
\begin{Highlighting}[]
 \NormalTok{V1 <- }\KeywordTok{sapply}\NormalTok{(}\DecValTok{1}\NormalTok{:}\KeywordTok{length}\NormalTok{(h_grid), function(h)\{}
      \NormalTok{delta * F[[h]] %*% V +  }\KeywordTok{profit}\NormalTok{(x_grid, h_grid[h]) }
    \NormalTok{\})}
    \CommentTok{# find havest, h that gives the maximum value}
    \NormalTok{out <- }\KeywordTok{sapply}\NormalTok{(}\DecValTok{1}\NormalTok{:gridsize, function(j)\{}
      \NormalTok{value <- }\KeywordTok{max}\NormalTok{(V1[j,], }\DataTypeTok{na.rm =} \NormalTok{T) }\CommentTok{# each col is a diff h, max over these}
      \NormalTok{index <- }\KeywordTok{which.max}\NormalTok{(V1[j,])  }\CommentTok{# store index so we can recover h's }
      \KeywordTok{c}\NormalTok{(value, index) }\CommentTok{# returns both profit value & index of optimal h.  }
    \NormalTok{\})}
    \CommentTok{# Sets V[t+1] = max_h V[t] at each possible state value, x}
    \NormalTok{V <- out[}\DecValTok{1}\NormalTok{,]                        }\CommentTok{# The new value-to-go}
    \NormalTok{D[,OptTime-time}\DecValTok{+1}\NormalTok{] <- out[}\DecValTok{2}\NormalTok{,]       }\CommentTok{# The index positions}
\end{Highlighting}
\end{Shaded}

\emph{Currently this shows the literal R code, should be adapted}

\subsubsection{Discussion on how we compare performance of policies}

\begin{itemize}
\item
  Replicate stochastic simulations
\item
  Sensitivity analysis (Figure 4).
\end{itemize}

\section{Results}

\subsection{Figure 1:}

\emph{Shows the inferred Gaussian Process compared to the true and
parametric models. Refer to the appendix for details on the GP
posteriors, etc.}

\begin{figure}[htbp]
\centering
\includegraphics{figure/gp_plot.pdf}
\caption{Graph of the inferred Gaussian process compared to the true
process and maximum-likelihood estimated process. Graph shows the
expected value for the function $f$ under each model. Two standard
deviations from the estimated Gaussian process covariance with (light
grey) and without (darker grey) measurement error are also shown. The
training data is also shown as black points. (The GP is conditioned on
0,0, shown as a pseudo-data point).}
\end{figure}

\subsection{Figure 2:}

\emph{The take-home message, showing that the GP is closest to the
optimal strategy, while the parametric methods are less accurate.
Visualizing the policy may be more useful for the technical reader, the
general audience may prefer Figure 3 showing all replicates of the
population collapse under the parametric model and not under the GP.}

\begin{figure}[htbp]
\centering
\includegraphics{figure/policies_plot.pdf}
\caption{The steady-state optimal policy (infinite boundary) calculated
under each model. Policies are shown in terms of target escapement,
$S_t$, as under models such as this a constant escapement policy is
expected to be optimal (Reed 1979).}
\end{figure}

\subsection{Figure 3:}

\emph{Figure 3 is a less abstract and more visceral visualization of the
take-home message, with the structurally inaccurate model leading
universally to a collapse of the fishery and very few profits, while the
Gaussian process performs nearly optimally. The parametric approach even
with the correct underlying structure does not perform optimally,
choosing in this case to under-fish (may need to show harvest dynamics
since that is not clear from the figure! Also isn't general, sometimes
does optimally, sometimes over-fishes. Perhaps need to show more
examples.) May need to show profits too?}

\begin{figure}[htbp]
\centering
\includegraphics{figure/sim_plot.pdf}
\caption{Gaussian process inference outperforms parametric estimates.
Shown are 100 replicate simulations of the stock dynamics (eq 1) under
the policies derived from each of the estimated models, as well as the
policy based on the exact underlying model.}
\end{figure}

\subsection{Figure 4:}

\emph{Shows the sensitivity analysis. A histogram of distribution of
yield over stochastic realizations, showing that the qualitative results
do not depend on the stochastic realization of the training data here,
or on the parameters of the underlying model, though quantitative
differences are visible.}

\section{Discussion / Conclusion}

\begin{itemize}
\item
  Non-parametric methods have received far too little attention in
  ecological modeling efforts that are aimed at improved conservation
  planning and decision making support.
\item
  Importance of non-parametric approaches in conservation planning /
  resource management / decision theory.
\item
  Decision-theoretic tools such as optimal control calculations rely on
  robust \emph{forecasting} more strongly than they rely on accurate
  \emph{mechanistic} relationships.
\item
  Adapting a non-parametric approach requires modification of existing
  methods for decision theory. We have illustrated how this might be
  done in the context of stochastic dynamic programming, opening the
  door for substantial further research into how these applications
  might be improved.
\item
  Anticipate improved relative performance in higher dimensional
  examples
\item
  Discuss constant escapement in model, in policies.
\item
  Limitations of this comparison: Are the maximum-likelihood solutions a
  straw man?
\item
  Discussion of alternative related approaches: POMDP/MOMDP,
\end{itemize}

\subsection{Future directions}

\begin{itemize}
\item
  Multiple species
\item
  Online learning
\item
  Multiple step-ahead predictions
\item
  Explicitly accomidating additional uncertainties
\item
  Improving inference of optimal policy from the GP
\end{itemize}

\section{Appendix / Supplementary Materials}

\subsection{MCMC posterior distributions and convergence analysis}

\begin{figure}[htbp]
\centering
\includegraphics{figure/posteriors.pdf}
\caption{Histogram of posterior distributions for the estimated Gaussian
Process shown in Figure 1. Prior distributions overlaid.}
\end{figure}

@Gramacy2005

\subsection{Tables of nuisance parameters, sensitivity analysis}

\subsubsection{List of hyper-parameters, prior distributions and their
parameters}

\subsection{Reproducible code, ``Research Compendium''}

Allen, Linda J. S., Jesse F. Fagan, Göran Högnäs, and Henrik Fagerholm.
2005. ``Population extinction in discrete-time stochastic population
models with an Allee effect.'' \emph{Journal of Difference Equations and
Applications} 11 (apr): 273--293. doi:10.1080/10236190412331335373.

Brozović, Nicholas, and Wolfram Schlenker. 2011. ``Optimal management of
an ecosystem with an unknown threshold.'' \emph{Ecological Economics}
(jan): 1--14. doi:10.1016/j.ecolecon.2010.10.001.

Courchamp, Franck, Ludek Berec, and Joanna Gascoigne. 2008. \emph{Allee
Effects in Ecology and Conservation}. Oxford University Press, USA.

Cressie, Noel, Catherine a Calder, James S. Clark, Jay M. Ver Hoef, and
Christopher K. Wikle. 2009. ``Accounting for uncertainty in ecological
analysis: the strengths and limitations of hierarchical statistical
modeling.'' \emph{Ecological Applications} 19 (apr): 553--70.

Geritz, Stefan a. H., and Eva Kisdi. 2011. ``Mathematical ecology: why
mechanistic models?.'' \emph{Journal of mathematical biology} (dec).
doi:10.1007/s00285-011-0496-3.

Levins, Richard. 1966. ``The strategy of model building in population
biology.'' \emph{American Scientist} 54: 421--431.

Mangel, Marc, and Colin W. Clark. 1988. \emph{Dynamic Modeling in
Behavioral Ecology}. Princeton: Princeton University Press.

Munch, Stephan B., Melissa L. Snover, George M. Watters, and Marc
Mangel. 2005. ``A unified treatment of top-down and bottom-up control of
reproduction in populations.'' \emph{Ecology Letters} 8 (may): 691--695.
doi:10.1111/j.1461-0248.2005.00766.x.

Polasky, Stephen, Stephen R. Carpenter, Carl Folke, and Bonnie Keeler.
2011. ``Decision-making under great uncertainty: environmental
management in an era of global change.'' \emph{Trends in ecology \&
evolution} (may): 1--7. doi:10.1016/j.tree.2011.04.007.

Reed, William J. 1979. ``Optimal escapement levels in stochastic and
deterministic harvesting models.'' \emph{Journal of Environmental
Economics and Management} 6 (dec): 350--363.
doi:10.1016/0095-0696(79)90014-7.

Scheffer, Marten, Stephen R. Carpenter, J. A. Foley, C. Folke, and B.
Walker. 2001. ``Catastrophic shifts in ecosystems.'' \emph{Nature} 413
(oct): 591--6. doi:10.1038/35098000.


\bibliography{}


\end{document}
