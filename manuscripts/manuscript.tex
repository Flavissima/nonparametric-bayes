\documentclass[author-year, 12pt,review]{components/elsarticle} %review=doublespace preprint=single 5p=2 column
%%% Begin My package additions %%%%%%%%%%%%%%%%%%%
\usepackage[hyphens]{url}
\usepackage{lineno} % add 
  \linenumbers % turns line numbering on 
\bibliographystyle{elsarticle-harv}
\biboptions{sort&compress} % For natbib
\usepackage{graphicx}
\usepackage{booktabs} % book-quality tables
%% Redefines the elsarticle footer
\makeatletter
\def\ps@pprintTitle{%
 \let\@oddhead\@empty
 \let\@evenhead\@empty
 \def\@oddfoot{\it \hfill\today}%
 \let\@evenfoot\@oddfoot}
\makeatother

% A modified page layout
\textwidth 6.75in
\oddsidemargin -0.15in
\evensidemargin -0.15in
\textheight 9in
\topmargin -0.5in
%%%%%%%%%%%%%%%% end my additions to header

\usepackage[T1]{fontenc}
\usepackage{lmodern}
\usepackage{amssymb,amsmath}
\usepackage{ifxetex,ifluatex}
\usepackage{fixltx2e} % provides \textsubscript
% use upquote if available, for straight quotes in verbatim environments
\IfFileExists{upquote.sty}{\usepackage{upquote}}{}
\ifnum 0\ifxetex 1\fi\ifluatex 1\fi=0 % if pdftex
  \usepackage[utf8]{inputenc}
\else % if luatex or xelatex
  \usepackage{fontspec}
  \ifxetex
    \usepackage{xltxtra,xunicode}
  \fi
  \defaultfontfeatures{Mapping=tex-text,Scale=MatchLowercase}
  \newcommand{\euro}{€}
\fi
% use microtype if available
\IfFileExists{microtype.sty}{\usepackage{microtype}}{}
\usepackage{longtable}
\usepackage{graphicx}
% We will generate all images so they have a width \maxwidth. This means
% that they will get their normal width if they fit onto the page, but
% are scaled down if they would overflow the margins.
\makeatletter
\def\maxwidth{\ifdim\Gin@nat@width>\linewidth\linewidth
\else\Gin@nat@width\fi}
\makeatother
\let\Oldincludegraphics\includegraphics
\renewcommand{\includegraphics}[1]{\Oldincludegraphics[width=\maxwidth]{#1}}
\ifxetex
  \usepackage[setpagesize=false, % page size defined by xetex
              unicode=false, % unicode breaks when used with xetex
              xetex]{hyperref}
\else
  \usepackage[unicode=true]{hyperref}
\fi
\hypersetup{breaklinks=true,
            bookmarks=true,
            pdfauthor={},
            pdftitle={Avoiding tipping points in fisheries management through Gaussian Process Dynamic Programming},
            colorlinks=true,
            urlcolor=blue,
            linkcolor=magenta,
            pdfborder={0 0 0}}
\urlstyle{same}  % don't use monospace font for urls
\setlength{\parindent}{0pt}
\setlength{\parskip}{6pt plus 2pt minus 1pt}
\setlength{\emergencystretch}{3em}  % prevent overfull lines
\setcounter{secnumdepth}{0}
% Pandoc toggle for numbering sections (defaults to be off)
\setcounter{secnumdepth}{0}
% Pandoc header



\begin{document}
\begin{frontmatter}

  \title{Avoiding tipping points in fisheries management through Gaussian Process
Dynamic Programming}
    \author[cstar]{Carl Boettiger\corref{c1}}
   \ead{cboettig(at)gmail.com} 
   \cortext[c1]{Corresponding author}
    \author[cstar]{Marc Mangel}
  
  
    \author[noaa]{Stephan Munch}
  
  
      \address[cstar]{Center for Stock Assessment Research, Department of Applied Math and
Statistics, University of California, Mail Stop SOE-2, Santa Cruz, CA
95064, USA}    
    \address[marc]{Center for Stock Assessment Research, Department of Applied Math and
Statistics, University of California, Mail Stop SOE-2, Santa Cruz, CA
95064, USA and Department of Biology, University of Bergen, Bergen,
Norway 9020}    
    \address[noaa]{Southwest Fisheries Science Center, National Oceanic and Atmospheric
Administration, 110 Shaffer Road, Santa Cruz, CA 95060, USA}    
  
  \begin{abstract}
  Model uncertainty and limited data are fundamental challenges to robust
  management of human intervention in a natural system. These challenges
  are acutely highlighted by concerns that many ecological systems may
  contain tipping points, such as Allee population sizes. Before a
  collapse, we do not know where the tipping points lie, if they exist at
  all. Hence, we know neither a complete model of the system dynamics nor
  do we have access to data in some large region of state-space where such
  a tipping point might exist. We illustrate how a Bayesian Non-Parametric
  (BNP) approach using a Gaussian Process (GP) prior provides a flexible
  representation of this inherent uncertainty. We embed GPs in a
  Stochastic Dynamic Programming (SDP) framework in order to make robust
  management predictions with both model uncertainty and limited data. We
  use simulations to evaluate this approach as compared with the standard
  approach of using model selection to choose from a set of candidate
  models. We find that model selection erroneously favors models without
  tipping points -- leading to harvest policies that guarantee extinction.
  The GPDP performs nearly as well as the true model and significantly
  outperforms standard approaches. We illustrate this using examples of
  simulated single-species dynamics, where the standard model selection
  approach should be most effective, and find that it still fails to
  account for uncertainty appropriately and leads to population crashes,
  while management based on the GPDP does not, since it does not
  underestimate the uncertainty outside of the observed data.
  \end{abstract}
   \begin{keyword} Bayesian \sep Structural Uncertainty \sep Nonparametric \sep Optimal Control \sep Decision Theory \sep Gaussian Processes \sep Fisheries Management \sep \end{keyword}
 \end{frontmatter}


\section{Introduction}\label{introduction}

Decision making under uncertainty is a ubiquitous challenge in the
management of human intervention in natural resources and conservation.
Ecological dynamics are frequently complex and difficult to measure,
making uncertainty in our understanding a persistent challenge.
Decision-theoretic approaches provide a framework to determine the best
sequence of actions in face of uncertainty, but only when that
uncertainty can be meaningfully quantified (Fischer et al. 2009). Over
the last four decades (but stretching even further back in time), marked
by the seminal contributions of Clark (1976), Clark (2009) and Walters
and Hilborn (1978), dynamic optimization methods, particularly
Stochastic Dynamic Programming (SDP), have become increasingly important
as a means of understanding how to manage human intervention into
natural systems. Simultaneously, there has been increasing recognition
of the importance of Allee population sizes or `tipping points'
(Scheffer et al. 2001, Polasky et al. 2011) that management actions may
inadvertently cause populations to cross. Scheffer et al. (2009)
emphasized the difficulties of formulating even qualitatively correct
models of the underlying processes.

We develop a novel approach to address these concerns in the context of
fisheries; though the underlying challenges and methods are germane to
many other problems of conservation or natural resource exploitation,
such as forestry. Economic value and ecological concern have made marine
fisheries the crucible for much of the founding work (Gordon 1954, Clark
1976, 2009, May et al. 1979, Reed 1979, Ludwig and Walters 1982) on
management under uncertainty. Global declines (Worm et al. 2006) and the
controversy surrounding their interpretation (Hilborn 2007, Worm et al.
2009) make understanding these challenges all the more pressing.

Uncertainty enters the decision-making process at many levels: intrinsic
stochasticity in biological processes, measurements, and implementation
of policy (\emph{e.g.} Reed 1979, Clark and Kirkwood 1986, Roughgarden
and Smith 1996, Sethi et al. 2005), parametric uncertainty (\emph{e.g.}
Ludwig and Walters 1982, Hilborn and Mangel 1997, McAllister 1998,
Schapaugh and Tyre 2013), and model or structural uncertainty
(\emph{e.g.} Williams 2001, Cressie et al. 2009, Athanassoglou and
Xepapadeas 2012). Of these, structural uncertainty is generally the
hardest to quantify. Typical approaches assume a weak notion of model
uncertainty in which the correct model (or reasonable approximation) of
the dynamics must be identified from among a handful of alternative
models, using either model choice, model averaging, or introducing yet
greater model complexity of which others may be special cases (model
averaging being one such way to construct such a model) (Williams 2001,
Cressie et al. 2009, Athanassoglou and Xepapadeas 2012). Even setting
aside other computational and statistical concerns (e.g. Cressie et al.
(2009)), these approaches do not address our second concern -
representing uncertainty outside the observed data range.

Model uncertainty is particularly insidious when model predictions must
be made outside of the range of data on which the model was estimated,
and models may be abused (Mangel et al. 2001). Extrapolation uncertainty
is felt most keenly in decision-theoretic (or optimal control)
applications, since (a) exploring the potential action space typically
involves considering actions that may move the system outside the range
of observed behavior, and (b) decision-theoretic algorithms rely not
only on reasonable estimates of the expected outcomes, but depend on the
weights given to all possible outcomes (\emph{e.g.} Weitzman 2013).

The dual concerns of model uncertainty and incomplete data coverage pose
a substantial challenge to existing decision-theoretic approaches
(Brozović and Schlenker 2011). Because intervention is often (but not
always, see Hughes et al. (2013)) too late after a tipping point has
been crossed, management is most often concerned with avoiding
potentially catastrophic tipping points before any data are available at
or following a transition that would more clearly reveal these regime
shift dynamics (e.g. Bestelmeyer et al. 2012).

We illustrate how a Stochastic Dynamic Programming (SDP) algorithm
(Mangel and Clark 1988, Marescot et al. 2013) can be driven by the
predictions from a Bayesian Non-Parametric (BNP) model of population
dynamics (Munch et al. 2005a). This provides two distinct advantages
compared with contemporary approaches. First, using a BNP sidesteps the
need for an accurate model-based description of the system dynamics.
Second, the BNP can better reflect uncertainty that arises when
extrapolating a model outside of the data on which it was fit. We
illustrate that when the correct model is not known, this latter feature
is crucial to providing a robust decision-theoretic approach in face of
substantial structural uncertainty.

This paper represents the first ecological application of the SDP
decision-making framework without an a priori model of the underlying
dynamics. In contrast to parametric models which can only reflect
uncertainty in parameter estimates, the BNP approach provides a
state-space dependent representation of uncertainty. This permits a much
greater uncertainty far from the observed data than near the observed
data. These features allow the Gaussian Process Dynamic Programming
(GPDP) approach to find robust management solutions in face of limited
data and without knowledge of the correct model structure.

To illustrate GPDP, we consider the performance of the GPDP policy
against the policies derived under several alternative parametric models
(Reed 1979, Ludwig and Walters 1982, Mangel and Clark 1988). The nature
of decision-making problems provides a compelling and pragmatic way to
compare models: Rather than compare models in terms of best fit to data,
we define model performance in the concrete terms of the
decision-maker's objectives.

\section{Approach and Methods}\label{approach-and-methods}

We first describe the requirements of dynamic optimization for the
management of human intervention in natural resource systems. After that
we describe three parametric models for population dynamics and the GP
description of population dynamics.

\subsubsection{Requirements of dynamic
optimization}\label{requirements-of-dynamic-optimization}

Dynamic optimization requires characterizing the dynamics of a state
variable (or variables), a control action, and a value function. In this
paper, we consider only a single state variable, so that the ideas can
be described as simply as possible. That is, we focus on a class of
one-dimensional models of population dynamics as a `best-case' scenario
for the parametric model performance relative to the GP. This is a
best-case scenario for the parametric models because the underlying
dynamics are always simulated from one of the parametric models in the
set, whereas in reality we never have the ``true'' model among our
candidates. By choosing one-dimensional models with few parameters, we
limit the chance that poor performance will be due to our inability to
estimate parameters well, something that becomes a more severe problem
for higher-dimensional parametric models. Further, the parametric models
we consider are those most commonly used in modeling stock-recruitment
dynamics or to model sudden transitions between alternative stable
states.

\subsubsection{Parametric models}\label{parametric-models}

We let $X_t$ denote the size (numbers or biomass) of the focal
population at time $t$ and assume that in the absence of take its
dynamics are.

\begin{equation}
X_{t+1}= Z_t f(X_t) \label{eq1} 
\end{equation}

Where Z(t) is log-normally distributed process stochasticity (Reed 1979)
and $p$ is a vector of parameters to be estimated from the data. We
describe three choices for $f(X_t,p)$ in the next section.In this simple
model, the control action is a harvest or take, $h_t$, measured in the
same units as $X$, at time $t$. Thus, in the presence of take, the
population size on the right hand side of Eqn 1 is replaced by
$S_t=X_t-h_t$.

To construct the value function, we consider a return when $X_t=x$ and
harvest $h_t=h$ denoted as the reward, $R(x,h)$. For example, if the
return is simply the harvest at time $t$, then $R(x,h)=min(x,h)$. We
assume that future harvests are discounted relative to current ones at a
constant rate of discount $\delta$ and ask for the harvest policy that
maximizes total discounted harvest between the current time $t$ and a
final time $T$. That is, we seek to maximize over choices of harvest
$\mathbf{E}_{X_t+1} [ \sum_{t = 0}^{T}  R_t(X_t, h_t) \delta^t]$, where
the state dynamics are given by Eqn 1 and $\mathbf{E}$ denotes the
expectation over the process stochasticity in the future population
state.

In order to find that policy, we introduce the value function $V_t(x_t)$
representing the total discounted catch from time $t$ onwards given that
$X_t=x_t$. This value function satisfies an equation of SDP (Mangel and
Clark 1988, Clark and Mangel 2000, Clark 2009), sometimes called the
Hamilton-Jacobi-Bellman equation (Mangel 2014),

\begin{equation}
V_t(x_t) = \max_{h_t} \lbrace R(h_t, x_t) + \delta \cdot \mathbf{\mathrm{E}}_{X_{t+1}} \left[ V_{t+1}( X_{t+1}) | x_t, h_t \right] \rbrace
\end{equation}

where expectation is taken over all possible values of the next state,
$X_{t+1}$, and maximized over all possible choices of harvest, $h_t$.
That is, at time $t$, when population size is $x_t$ and harvest $h_t$ is
applied, the immediate return is $R(h_t, x_t)$. In the classical case,
the sole source of uncertainty is the process stochasticity term, $Z$,
and thus the expectation above is equivalent to taking expectations over
$Z$. That is

\[ \mathbf{\mathrm{E}}_{X_{t+1}} \left[ V_{t+1}( X_{t+1}) | x_t, h_t \right] = \mathbf{\mathrm{E}}_{Z} \left[ V_{t+1}( Z f(x_t - h_t))  | x_t, h_t \right] \]

where the population size after the take is $X_t-h_t$, which is then
translated into $X_{t+1}$ by Eqn 1.

In a Bayesian decision framework, the parameters governing the dynamics
are also uncertain, so the expectation involves averaging over the
posterior distribution for the parameters, as well. That is,

\[\mathbf{\mathrm{E}}_{X_t+1} \left[ V_{t+1}( X_{t+1}) | x_t, h_t \right] = \mathbf{\mathrm{E}}_{\theta|\mathrm{data}} \{ \mathbf{\mathrm{E}}_{Z | \theta, \mathrm{data}} \left[ V_{t+1}( Z f(x_t - h_t))  | x_t, h_t \right] \}\]

In the nonparametric case, the function $f$ too is uncertain and the
expectation for the next state includes uncertainty in $f$ as well. That
is

\[\mathbf{\mathrm{E}}_{X_t+1} \left[ V_{t+1}( X_{t+1}) | X_t, h_t \right] = \mathbf{\mathrm{E}}_{\theta|\mathrm{data}} \{ \mathbf{\mathrm{E}}_{f, Z | \theta, \mathrm{data}} \left[ V_{t+1}( Z f(X_t - h_t))  | x_t, h_t \right] \}\]

We consider the finite time problem with $T=$ 1000, which we solve using
the standard value iteration algorithm (see Mangel and Clark 1988, Clark
and Mangel 2000).

\subsubsection{Parametric models}\label{parametric-models}

We consider three candidate parametric models of the stock-recruitment
dynamics: The Ricker model, the Allen model (Allen and Tanner 2005), the
Myers model (Myers et al. 1995), Eqns \eqref{ricker}-\eqref{myers},
which we parameterize as follows for $f(S_t,p)$ in Eqn 1. In all three,
we let $K$ denote the carrying capacity and $r$ the maximum per capita
growth rate. In the Ricker model,

\begin{equation}
f(S(t)|p) = S_t e^{r \left(1 - \frac{S_t}{K} \right) } \label{ricker}
\end{equation}

In the Allen model,

\begin{equation}
f(S(t)|p) = S_t e^{r \left(1 - \frac{S_t}{K}\right)\left(S_t - X_c\right)} \label{allen}
\end{equation}

Where $r$ and $K$ are as before, and $X_c$ denotes the location of the
unstable steady state (i.e., the tipping point).

In the Myers model,

\begin{equation}
f(S(t) | p)  = \frac{r S_t^{\theta}}{1 + \frac{S_t^\theta}{K}} \label{myers}
\end{equation}

Where $\theta = 1$ corresponding to Beverton-Holt dynamics and $\theta$
\textgreater{} 2 leads to Allee effects and multiple stable states.

The Ricker model involves two parameters, corresponding to a growth rate
and a carrying capacity, and cannot support alternative stable growth
rate dynamics. The Allen model resembles the Ricker dynamics with an
added Allee effect parameter (Courchamp et al. 2008), below which the
population cannot persist. The Myers model also has three parameters and
contains an Allee threshold, but has compensatory rather than
over-compensatory density dependence (resembling a Beverton-Holt curve
rather than a Ricker curve at high densities.)

The multiplicative log-normal stochasticity perturbs the growth
predicted by each of these deterministic model skeletons. This
introduces one additional parameter $\sigma$ that must be estimated.

Because of our interest in management performance in the presence of
tipping points, all of our simulations are based on the Allen model. The
Allen model is thus ``structurally correct'' and is expected to provide
a best-case scenario when the `true' dynamics are known. The Ricker
model is a reasonable approximation of these dynamics far from the Allee
threshold (but lacks threshold dynamics), while the Myers model shares
the essential feature of a threshold but differs in the structure.

We consider a period of 40 years of training data: long enough that the
estimates do not depend on the particular realization, while longer
times are not likely to provide substantial improvement.\\Each of the
models is fit to the same training data (Figure 1).

We inferred posterior distributions for the parameters of each model by
Gibbs sampling (Gelman et al. (2003) implemented in R (R Core Team 2013)
using \texttt{jags}, (Su and Masanao Yajima 2013)). We choose uniform
priors for all parameters (See appendix Tables S1-S3; R code provided).
We show one-step-ahead predictions of these model fits in Figure 1. We
tested each chain for Gelman-Rubin convergence and results were robust
to longer runs. For each simulation we also applied several commonly
used model selection criteria (AIC, BIC, DIC, see Burnham and Anderson
(2002)) to identify the best fitting model.

\subsubsection{The Gaussian Process
model}\label{the-gaussian-process-model}

The core difference for our purpose between the GP representation and
the parametric models is that the GP model is defined explicitly in
reference to some observed data. Unlike the models above, it cannot be
specified by the value of some parameters alone. Although called
nonparametric, the GP (a) still involves the estimation of parameters,
and (b), statisticians also use non-parametric to mean models that make
only weak distributional assumptions (e.g. Lehmann 1975). However, we
retain the terminology for consistency with previous studies.

The use of GP methods to formulate a predictive model is relatively new
in the context of modeling dynamical systems (Kocijan et al. 2005), and
was first introduced in the context ecological modeling and fisheries
management in Munch et al. (2005b). GP models have subsequently been
used to test for the presence of Allee effects (Sugeno and Munch 2013a),
estimate the maximum reproductive rate (Sugeno and Munch 2013b),
determine temporal variation in food availability (Sigourney et al.
2012), and provide a basis for identifying model-misspecification
(Thorson et al. 2014). An accessible and thorough introduction to the
formulation and use of GPs can be found in Rasmussen and Williams
(2006).

A GP is an infinite dimensional (because it is a function of a
continuous input variable) stochastic process for a finite sample of
points from any realization is a multivariate normal distribution. Thus,
to characterize the GP we need a mean function and a covariance
function. We proceed as follows.

We assume that the data $X_o$ are observed with some process
stochasticity,

\[X_{t+1} = f(X_t) + \varepsilon,\]

where $\varepsilon$ are IID normal random variables with zero-mean and
variance $\sigma_n^2$. Note that we have chosen to assume additive
stochasticity. While we could just as easily consider log-normal
stochasticity as in the parametric models, we make this choice to bring
home the point that the Gaussian process approach need not have
structurally correct noise form either.

Then under the GP, given a vector of observed data $X_0$ the predicted
data $X_p$ obeys

\[f(X_p|X_o) \sim \mathcal{N}(E,C)\]
\[E = K(X_p, X_o) \left(K(X_o,X_o) - \sigma \mathbb{I} \right)  ^{-1} y\]
\[C = K(X_p, X_p) - K(X_p, X_o) K(X_o,X_o)^{-1} K(X_o, X_p)\]

Where $K$ is the Gaussian process kernel. (Compare to Eqn 2.22 of
Rasmussen and Williams (2006) for a more detailed derivation.)
Throughout, we use the radial basis function kernel:

\[ K(x,y) = \exp\left(\frac{-(x-y)^2}{2 \ell^2} \right)\]

We use inverse Gamma priors on both the length-scale $\ell$ and
$\sigma$, of the form

\[f(x; \alpha, \beta) = \frac{\beta^\alpha}{\Gamma(\alpha)} x^{-\alpha - 1}\exp\left(-\frac{\beta}{x}\right)\]

For the $\sigma$ prior, $\alpha = $ 5 and $\beta = $ 5. For $\ell$
prior, $\alpha = $ 10 and $\beta = $ 10.

Using the simulated training data we also estimate a Gaussian process
defined by a radial basis function kernel of two parameters: $\ell$,
which gives the characteristic length-scale over which correlation
between two points in state-space decays, and $\sigma$, which gives the
scale of the process noise by which observations $Y_{t+1}$ may differ
from their predicted values $X_{t+1}$ given an observation of the
previous state, $X_t$. Munch et al. (2005a) give an accessible
introduction to the use of Gaussian processes in providing a Bayesian
nonparametric description of the stock-recruitment relationship.

We use a Metropolis-Hastings Markov Chain Monte Carlo (Gelman et al.
(2003)) to infer posterior distributions of the two parameters of the GP
(Figure S13, code in appendix). Since the posterior distributions differ
substantially from the priors (Figure S13), we can be assured that most
of the information in the posterior comes from the data rather than the
prior belief.

\subsection{The method of Gaussian Process Dynamic Programming
(GPDP)}\label{the-method-of-gaussian-process-dynamic-programming-gpdp}

It is reasonably straight-forward to derive the harvest policy from the
estimated BP by inserting it into a SDP algorithm. The only difference
is that the uncertainty in the future state under the GP, $X_{t+1}$,
includes both process uncertainty (based on the estimation of $\sigma$)
and ``structural uncertainty'' of the posterior collection of
curves.\\(Recall that given the parameters, a parametric model is
represented by a single curve, while the GP remains a distribution of
curves).\\From the GP posteriors, we can write down the (discritized)
transition matrix representing the probability of going to each state
$X_{t+1}$ given any current state $X_t$ and any harvest $h_t$ (See the
function \texttt{gp\_transition\_matrix()} in the provided R package).
Given this transition matrix, we use the same value iteration algorithm
as in the parametric case to determine the optimal policy.

\section{Results}\label{results}

\subsection{Parametric and GP models for population
dynamics}\label{parametric-and-gp-models-for-population-dynamics}

To ensure our results are robust to our choice of parameters, consider
96 different scenarios, described in detail below. To help better
understand the process, we first describe in detail the results of a
single scenario.

\begin{figure}[htbp]
\centering
\includegraphics{components/figure/manuscript-figure_1.pdf}
\caption{Points show the training data of stock-size over time. Curves
show the posterior step-ahead predictions based on each of the estimated
models. Observe that all models are fitting the data reasonably well.}
\end{figure}

All models fit the observed data rather closely and with relatively
small uncertainty, as illustrated in the posterior predictive curves in
Figure 1, in which we show the training data of stock sizes observed
over time as points, overlaid with the step-ahead predictions of each
estimated model using the parameters sampled from their posterior
distributions. Each model manages to fit the observed data rather
closely. Compared to the true model most estimates appear to over-fit,
predicting fluctuations that are actually due purely to stochasticity.
Model selection criteria (Table 1) penalize more complex models and show
a preference for the simpler Ricker model over the models with
alternative stable states (Allen and Myers). Details on MCMC estimates
for each model, traces, and posterior distributions can be found in the
appendix.

\begin{longtable}[c]{@{}cccc@{}}
\toprule\addlinespace
\begin{minipage}[b]{0.12\columnwidth}\centering
~
\end{minipage} & \begin{minipage}[b]{0.10\columnwidth}\centering
Allen
\end{minipage} & \begin{minipage}[b]{0.11\columnwidth}\centering
Ricker
\end{minipage} & \begin{minipage}[b]{0.11\columnwidth}\centering
Myers
\end{minipage}
\\\addlinespace
\midrule\endhead
\begin{minipage}[t]{0.12\columnwidth}\centering
\textbf{DIC}
\end{minipage} & \begin{minipage}[t]{0.10\columnwidth}\centering
50.75
\end{minipage} & \begin{minipage}[t]{0.11\columnwidth}\centering
50.45
\end{minipage} & \begin{minipage}[t]{0.11\columnwidth}\centering
50.41
\end{minipage}
\\\addlinespace
\begin{minipage}[t]{0.12\columnwidth}\centering
\textbf{AIC}
\end{minipage} & \begin{minipage}[t]{0.10\columnwidth}\centering
-24.51
\end{minipage} & \begin{minipage}[t]{0.11\columnwidth}\centering
-30.13
\end{minipage} & \begin{minipage}[t]{0.11\columnwidth}\centering
-27.01
\end{minipage}
\\\addlinespace
\begin{minipage}[t]{0.12\columnwidth}\centering
\textbf{BIC}
\end{minipage} & \begin{minipage}[t]{0.10\columnwidth}\centering
-17.75
\end{minipage} & \begin{minipage}[t]{0.11\columnwidth}\centering
-25.06
\end{minipage} & \begin{minipage}[t]{0.11\columnwidth}\centering
-20.25
\end{minipage}
\\\addlinespace
\bottomrule
\addlinespace
\caption{Model selection scores for several common criteria all
(wrongly) select the simplest model. As the true (Allen) model is not
distinguishable from the simpler (Ricker) model in the region of the
observed data, this error cannot be avoided regardless of the model
choice criterion. This highlights the danger of model choice when the
selected model will be used outside of the observed range of the data.}
\end{longtable}

\begin{figure}[htbp]
\centering
\includegraphics{components/figure/manuscript-figure_2.pdf}
\caption{The inferred Gaussian process compared to the true process and
maximum-likelihood estimated process. We show the expected value for the
function $f$ under each model. Two standard deviations from the
estimated Gaussian process covariance with (light grey) and without
(darker grey) measurement error are also shown. The training data are
also shown as black points. The GP is conditioned on (0,0), shown as a
pseudo-data point.}
\end{figure}

We show the mean inferred state space dynamics of each model relative to
the true model used to generate the data in Figure 2, predicting the
relationship between observed stock size (x-axis) to the stock size
after recruitment the following year. Note that in contrast to the other
models shown, the expected Gaussian process corresponds to a
distribution of curves - as indicated by the gray band - which itself
has a mean shown in black. Parameter uncertainty (not shown) spreads out
the estimates further. The observed data from which each model are
estimated is also shown. The observations come from only a limited
region of state space corresponding to unharvested or weakly harvested
system. No observations occur at the theoretical optimum harvest rate or
near the tipping point.

Figure 3 illustrates the performance of the GP and parametric models
outside the observed training data. The mean trajectory under the
underlying model is shown by the black dots, while the corresponding
prediction made by the model shown by the box and whiskers plots.
Predictions are based on the true expected value in the previous time
step. Predicted distributions that lie entirely above the expected
dynamics indicate the expectation of stock sizes higher than what is
actually expected. The models differ both in their expectations and
their uncertainty (colored bands show two standard deviations away).
Note that the GP is particularly uncertain about the dynamics relative
to structurally incorrect models like the Ricker.

\begin{figure}[htbp]
\centering
\includegraphics{components/figure/manuscript-figure_3.pdf}
\caption{Outside the range of the training data (Figure 1), the true
dynamics (black dots) fall outside the uncertainty (two standard
deviations, colored bands) of the structurally incorrect parametric
models (Ricker, Myers), but inside the uncertainty predicted by the GP.
Points show the stock size simulated by the true model. Overlay shows
the range of states predicted by each model, based on the state observed
in the previous time step. The Ricker model always (wrongly) predicts
positive population growth, while the actual population shrinks in each
step as the initial condition falls below the Allee threshold of the
underlying model (Allen). Note that because it does not assume a
parametric form but instead relies more directly on the data, the GP is
both more pessimistic and more uncertain about the future state than the
parametric models.}
\end{figure}

\begin{figure}[htbp]
\centering
\includegraphics{components/figure/manuscript-figure_4.pdf}
\caption{The steady-state optimal policy (infinite boundary) calculated
under each model. Policies are shown in terms of target escapement,
$S_t$, as under models such as this a constant escapement policy is
expected to be optimal (Reed 1979).}
\end{figure}

Despite the similarities in model fits to the observed data, the
policies inferred under each model differ widely (Figure 4). Policies
are shown in terms of target escapement, $S_t$. Under models such as
this a constant escapement policy is expected to be optimal (Reed 1979),
whereby population levels below a certain size $S$ are unharvested,
while above that size the harvest strategy aims to return the population
to $S$, resulting in the hockey-stick shaped policies shown. Only the
structurally correct model (Allen model) and the GP produce policies
close to the true optimum policy (where both the underlying model
structure and parameter values are known without error).

\begin{figure}[htbp]
\centering
\includegraphics{components/figure/manuscript-figure_5.pdf}
\caption{In the management context, GPDP outperforms approaches based on
parametric models. We show 100 replicate simulations of the stock
dynamics (eq 1) under the policies derived from each of the estimated
models, as well as the policy based on the exact underlying model.}
\end{figure}

In Figure 5, we show the consequences of managing 100 replicate
realizations of the simulated fishery under each of the policies
estimated. As anticipated from the policy curves, the structurally
correct model under-harvests, leaving the stock to vary around its
unfished optimum. The structurally incorrect Ricker model over-harvests
the population passed the tipping point consistently, resulting in the
immediate crash of the stock and thus derives minimal long-term catch.

The results across replicate stochastic simulations can most easily be
compared by using the relative differences in net present value realized
by each of the model (Figure 6). The GPDP most consistently realizes a
value close to the optimal solution, and importantly avoids ever driving
the system across the tipping point, which results in the near-zero
value cases in the parametric models.

\begin{figure}[htbp]
\centering
\includegraphics{components/figure/manuscript-figure_6.pdf}
\caption{Histograms of the realized net present value of the fishery
over a range of simulated data and resulting parameter estimates. For
each data set, the three models are estimated as described above. Values
plotted are the averages of a given policy over 100 replicate
simulations. Details and code provided in the supplement.}
\end{figure}

\subsection{Sensitivity Analysis}\label{sensitivity-analysis}

These results are not sensitive to the modeling details of the
simulation. The GPDP estimate remains very close to the optimal solution
obtained by knowing the true model across changes to the training
simulation, noise scale, parameters or structure of the underlying
model. We consider both a Latin hypercube approach and a more focused
investigation of the parameters most responsible for impacting the
transition rate: the relative distance to the Allee threshold and the
noise effects.

The Latin hypercube approach systematically varies all combinations of
parameters, providing a more general test than varying only one
parameter at a time. We loop across eight replicates of three different
randomly generated parameter sets for each of two different generating
models (Allen and Myers) over two different noise levels (0.01 and
0.05), for a total of 8 x 3 x 2 x 2 = 96 scenarios. The Gaussian Process
performs nearly optimally in each case, relative to the optimal solution
with no parameter or model uncertainty (Figure S9, appendix).

Changing the intensity of the stochasticity or the distance between
stable and unstable points does not impact the performance of the GP
relative to the optimal solution given the true model and true
parameters. The parametric models are more sensitive to this difference.
Large noise relative to the distance between the stable and unstable
point increases the chance of a stochastic transition. More precisely,
if we let $L = K - x_c$, then the probability of a transition in a given
window of time $T$ scales as

\[P(x_t < x_c | t \in T) \propto  \exp\left(\frac{L^2}{\sigma^2}\right),\]

(see Gardiner (2009) or Mangel (2006) for the derivation). As a result,
the impact of using a model which underestimates the risk of harvesting
past the critical point is much more severe, since this such a situation
occurs more often. Conversely, with large enough distance between the
optimal escapement and unstable points relative to the noise, the chance
of a transition becomes vanishingly small and all models can be
estimated near-optimally. Models that underestimate the cost incurred by
stock sizes fluctuating significantly below the optimal escapement level
will not perform poorly as long as those fluctuations are sufficiently
rare.

The GPDP is weakly influenced by increasing stochasticity or increasing
Allee effects over much of the range (Figure 7) Much larger
stochasticity or higher Allee effects make even the optimal solution
without any model or parameter uncertainty unable to harvest the
population effectively (e.g.~noise large enough to violate the
self-sustaining criterion of Reed (1979)). Only at the largest Allee
effect sizes are some replicates seen to deviate more from the optimal
solution.

\begin{figure}[htbp]
\centering
\includegraphics{components/figure/manuscript-figure_7.pdf}
\caption{The effect of increasing noise or decreasing Allee threshold
levels on the net present value of the fishery when managed under the
GPDP, relative to managing under the true model (with known parameters).
Other than the focal parameter (stochasticity, Allee threshold), other
parameters are held fixed as above to illustrate this effect.}
\end{figure}

\section{Discussion}\label{discussion}

Though simple mechanistically motivated models offer the greatest
potential to increase our basic understanding of ecological processes
(Geritz and Kisdi 2012, Cuddington et al. 2013), such models can be both
inaccurate and misleading when relied upon in a quantitative decision
making framework. In this paper we have tackled two aspects of
uncertainty that are both common to many ecological decision-making
problems and fundamentally challenging to existing approaches which
largely rely on parametric models:

\begin{enumerate}
\def\labelenumi{\arabic{enumi}.}
\itemsep1pt\parskip0pt\parsep0pt
\item
  We do not know what the correct models are for ecological systems.
\item
  We have limited data from which to estimate the model -- in
  particular, such models may be misleading in predicting the
  probability of outcomes outside the training data.
\end{enumerate}

We have illustrated how the use of non-parametric approaches can provide
more reliable solutions in the sequential decision-making problem.

\subsubsection{Traditional model-choice approaches can be positively
misleading.}\label{traditional-model-choice-approaches-can-be-positively-misleading.}

These results illustrate that model-choice approaches would be
positively misleading -- supporting simpler models that cannot express
tipping point dynamics merely on account of them being similar. As the
data shown comes only from the basin of attraction near the unfished
equilibrium, near which all of the models are approximately linear and
approximately identical.

Model selection criteria trade off model complexity and fit to the data.
When the data come from a limited region of state-space -- as is
necessarily the case whenever there is a potential concern about tipping
point dynamics -- simpler models therefore fit just as well and tend to
be selected more often than complex ones. This approach would be
appropriate when the dynamics can be expected to remain in the region of
the training data; for instance, if we only considered the forecasting
accuracy of the unfished population dynamics under each model.

In contrast, the decision-maker's problem of setting appropriate harvest
levels cannot exclude regions of state-space outside the observed range
when integrating over all possible decisions to find the optimal choice.
Such problems are not constrained to fisheries management but ubiquitous
across ecological decision-making and conservation where the greatest
concerns involve entering previously unobserved regions of state-space
-- whether that is the collapse of a fishery, the spread of an invasive,
or the loss of habitat.

\subsubsection{GPDP population dynamics capture larger uncertainty in
regions where the data are
poor}\label{gpdp-population-dynamics-capture-larger-uncertainty-in-regions-where-the-data-are-poor}

The parametric models perform worst when they propose a management
strategy outside the range of the observed data. The non-parametric
Bayesian approach, in contrast, allows a predictive model that expresses
a great deal of uncertainty about the probable dynamics outside the
observed range, while retaining very good predictive accuracy in the
range observed. The management policy dictated by the GP balances this
uncertainty against the immediate value of the harvest, and act to
stabilize the population dynamics in a region of state space in which
the predictions can be reliably reflected by the data.

\subsubsection{The role of the prior}\label{the-role-of-the-prior}

Lastly, it should be noted that outside the data, the GP reverts to the
prior, and consequently the choice of the prior can also play a
significant role in determining the optimal policy inferred by the SDP.
In the examples shown here we have selected a prior that is both
relatively uninformative (due to the broad priors placed on its
parameters $\ell$ and $\sigma$ and simple (mean zero, radial basis
function kernel). In practice, both the choice of mean and the
covariance function may be chosen to confer particular biological
properties, as well as more biologically informed priors for $\ell$ and
$\sigma$. In principle, this may allow a manager to improve the
performance of the GPDP by adding only enough additional detail as is
justified. For instance, it would be possible to use a linear or a
Ricker-shaped mean in the prior without making the much stronger
assumption that the Ricker is the structurally correct model (Sugeno and
Munch 2013a). Future research should focus on identifying criteria to
ensure the prior and the value function are chosen appropriately for the
problem at hand.

\section{Acknowledgments}\label{acknowledgments}

This work was partially supported by NOAA-IAM grant to SM and Alec
McCall and administered through the Center for Stock Assessment
Research, a partnership between the University of California Santa Cruz
and the Fisheries Ecology Division, Southwest Fisheries Science Center,
Santa Cruz, CA and by NSF grant EF-0924195 to MM and NSF grant
DBI-1306697 to CB.

Allen, D., and K. Tanner. 2005. Infusing active learning into the
large-enrollment biology class: seven strategies, from the simple to
complex. Cell biology education 4:262--8.

Athanassoglou, S., and A. Xepapadeas. 2012. Pollution control with
uncertain stock dynamics: When, and how, to be precautious. Journal of
Environmental Economics and Management 63:304--320.

Bestelmeyer, B. T., M. C. Duniway, D. K. James, L. M. Burkett, and K. M.
Havstad. 2012. A test of critical thresholds and their indicators in a
desertification-prone ecosystem: more resilience than we thought.
Ecology Letters:n/a--n/a.

Brozović, N., and W. Schlenker. 2011. Optimal management of an ecosystem
with an unknown threshold. Ecological Economics:1--14.

Burnham, K. P., and D. R. Anderson. 2002. Model Selection and
Multi-Model Inference. Page 496. Springer.

Clark, C. W. 1976. Mathematical Bioeconomics. WileyNew York.

Clark, C. W. 2009. Mathematical Bioeconomics. WileyNew York.

Clark, C. W., and G. P. Kirkwood. 1986. On uncertain renewable resource
stocks: Optimal harvest policies and the value of stock surveys. Journal
of Environmental Economics and Management 13:235--244.

Clark, C. W., and M. Mangel. 2000. Dynamic state variable models in
ecology. Oxford University PressOxford.

Courchamp, F., L. Berec, and J. Gascoigne. 2008. Allee Effects in
Ecology and Conservation. Page 256. Oxford University Press, USA.

Cressie, N., C. a Calder, J. S. Clark, J. M. {Ver Hoef}, and C. K.
Wikle. 2009. Accounting for uncertainty in ecological analysis: the
strengths and limitations of hierarchical statistical modeling.
Ecological Applications 19:553--70.

Cuddington, K. M., M. Fortin, and L. Gerber. 2013. Process-based models
are required to manage ecological systems in a changing world. Ecosphere
4:1--12.

Fischer, J., G. D. Peterson, T. a Gardner, L. J. Gordon, I. Fazey, T.
Elmqvist, A. Felton, C. Folke, and S. Dovers. 2009. Integrating
resilience thinking and optimisation for conservation. Trends in ecology
\& evolution 24:549--54.

Gardiner, C. 2009. Stochastic Methods: A Handbook for the Natural and
Social Sciences (Springer Series in Synergetics). Page 447. Springer.

Gelman, A., J. B. Carlin, H. S. Stern, and D. B. Rubin. 2003. Bayesian
Data Analysis. 2nd editions. Chapman; Hall/CRC.

Geritz, S. A. H., and E. Kisdi. 2012. Mathematical ecology: why
mechanistic models? Journal of mathematical biology 65:1411--5.

Gordon, H. 1954. The economic theory of a common-property resource: the
fishery. The Journal of Political Economy 62:124--142.

Hilborn, R. 2007. Reinterpreting the State of Fisheries and their
Management. Ecosystems 10:1362--1369.

Hilborn, R., and M. Mangel. 1997. The Ecological Detective: Confronting
Models with data. Page 330. Princeton University Press.

Hughes, T. P., C. Linares, V. Dakos, I. a van de Leemput, and E. H. van
Nes. 2013. Living dangerously on borrowed time during slow, unrecognized
regime shifts. Trends in ecology \& evolution 28:149--55.

Kocijan, J., A. Girard, B. Banko, and R. Murray-Smith. 2005. Dynamic
systems identification with Gaussian processes. Mathematical and
Computer Modelling of Dynamical Systems 11:411--424.

Lehmann, E. L. 1975. Nonparametrics: Statistical Methods Based on Ranks.
Holden-Day, IncSan Francisco.

Ludwig, D., and C. J. Walters. 1982. Optimal harvesting with imprecise
parameter estimates. Ecological Modelling 14:273--292.

Mangel, M. 2006. The Theoretical Biologist's Toolbox: Quantitative
Methods for Ecology and Evolutionary Biology. Cambridge University
Press.

Mangel, M. 2014. Stochastic Dynamic Programming Illuminates the Link
Between Environment. Bulletin of Mathematical Biology in press.

Mangel, M., and C. W. Clark. 1988. Dynamic Modeling in Behavioral
Ecology. (J. Krebs and T. Clutton-Brock, Eds.). Princeton University
PressPrinceton.

Mangel, M., 0. Fiksen, and J. Giske. 2001. Theoretical and statistical
models in natural resource management and research. Pages 57--71
\emph{in} T. M. Shenk and A. B. Franklin, editors. Modeling in natural
resource management, development, interpretation and application. Island
PressWashington DC.

Marescot, L., G. Chapron, I. Chadès, P. L. Fackler, C. Duchamp, E.
Marboutin, and O. Gimenez. 2013. Complex decisions made simple: a primer
on stochastic dynamic programming. Methods in Ecology and
Evolution:n/a--n/a.

May, R. M., J. R. Beddington, C. W. Clark, S. J. Holt, and R. M. Laws.
1979. Management of multispecies fisheries. Science (New York, N.Y.)
205:267--77.

McAllister, M. 1998. Bayesian stock assessment: a review and example
application using the logistic model. ICES Journal of Marine Science
55:1031--1060.

Munch, S. B., A. Kottas, and M. Mangel. 2005a. Bayesian nonparametric
analysis of stock-recruitment relationships. Canadian Journal of
Fisheries and Aquatic Sciences 62:1808--1821.

Munch, S. B., M. L. Snover, G. M. Watters, and M. Mangel. 2005b. A
unified treatment of top-down and bottom-up control of reproduction in
populations. Ecology Letters 8:691--695.

Myers, R. A., N. J. Barrowman, J. A. Hutchings, and A. a Rosenberg.
1995. Population dynamics of exploited fish stocks at low population
levels. Science (New York, N.Y.) 269:1106--8.

Polasky, S., S. R. Carpenter, C. Folke, and B. Keeler. 2011.
Decision-making under great uncertainty: environmental management in an
era of global change. Trends in ecology \& evolution:1--7.

R Core Team. 2013. R: A Language and Environment for Statistical
Computing. R Foundation for Statistical ComputingVienna, Austria.

Rasmussen, C. E., and C. K. I. Williams. 2006. Gaussian Processes for
Machine Learning. (Thomas Dietterich, Ed.). MIT Press,Boston.

Reed, W. J. 1979. Optimal escapement levels in stochastic and
deterministic harvesting models. Journal of Environmental Economics and
Management 6:350--363.

Roughgarden, J. E., and F. Smith. 1996. Why fisheries collapse and what
to do about it. Proceedings of the National Academy of Sciences of the
United States of America 93:5078.

Schapaugh, A. W., and A. J. Tyre. 2013. Accounting for parametric
uncertainty in Markov decision processes. Ecological Modelling
254:15--21.

Scheffer, M., J. Bascompte, W. A. Brock, V. Brovkin, S. R. Carpenter, V.
Dakos, H. Held, E. H. van Nes, M. Rietkerk, and G. Sugihara. 2009.
Early-warning signals for critical transitions. Nature 461:53--9.

Scheffer, M., S. R. Carpenter, J. A. Foley, C. Folke, and B. Walker.
2001. Catastrophic shifts in ecosystems. Nature 413:591--6.

Sethi, G., C. Costello, A. Fisher, M. Hanemann, and L. Karp. 2005.
Fishery management under multiple uncertainty. Journal of Environmental
Economics and Management 50:300--318.

Sigourney, D. B., S. B. Munch, and B. H. Letcher. 2012. Combining a
Bayesian nonparametric method with a hierarchical framework to estimate
individual and temporal variation in growth. Ecological Modelling
247:125--134.

Su, Y.-S., and Masanao Yajima. 2013. R2jags: A Package for Running jags
from R.

Sugeno, M., and S. B. Munch. 2013a. A semiparametric Bayesian method for
detecting Allee effects. Ecology 94:1196--1204.

Sugeno, M., and S. B. Munch. 2013b. A semiparametric Bayesian approach
to estimating maximum reproductive rates at low population sizes.
Ecological applications : a publication of the Ecological Society of
America 23:699--709.

Thorson, J. T., K. Ono, and S. B. Munch. 2014. A Bayesian approach to
identifying and compensating for model misspecification in population
models. Ecology 95:329--41.

Walters, C. J., and R. Hilborn. 1978. Ecological Optimization and
Adaptive Management. Annual Review of Ecology and Systematics
9:157--188.

Weitzman, M. L. 2013. A Precautionary Tale of Uncertain Tail Fattening.
Environmental and Resource Economics 55:159--173.

Williams, B. K. 2001. Uncertainty , learning , and the optimal
management of wildlife. Environmental and Ecological Statistics
8:269--288.

Worm, B., E. B. Barbier, N. Beaumont, J. E. Duffy, C. Folke, B. S.
Halpern, J. B. C. Jackson, H. K. Lotze, F. Micheli, S. R. Palumbi, E.
Sala, K. a Selkoe, J. J. Stachowicz, and R. Watson. 2006. Impacts of
biodiversity loss on ocean ecosystem services. Science (New York, N.Y.)
314:787--90.

Worm, B., R. Hilborn, J. K. Baum, T. A. Branch, J. S. Collie, C.
Costello, M. J. Fogarty, E. a Fulton, J. a Hutchings, S. Jennings, O. P.
Jensen, H. K. Lotze, P. M. Mace, T. R. McClanahan, C. Minto, S. R.
Palumbi, A. M. Parma, D. Ricard, A. a Rosenberg, R. Watson, and D.
Zeller. 2009. Rebuilding global fisheries. Science (New York, N.Y.)
325:578--85.

\end{document}


